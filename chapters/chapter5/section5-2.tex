\section{Hướng phát triển}

Mặc dù hệ thống đã đạt được những kết quả nhất định, tuy nhiên vẫn còn một số hạn chế cần được cải thiện trong tương lai. Một số hướng phát triển có thể được đề xuất như sau:

\begin{itemize}
    \item Nâng cấp và tối ưu thuật toán lộ trình để xử lý tốt hơn với tập dữ liệu lớn và điều kiện giao thông thay đổi theo thời gian thực.
    \item Tích hợp bản đồ và định vị thời gian thực nhằm hỗ trợ theo dõi vị trí đơn hàng chính xác hơn.
    \item Mở rộng hệ thống thông báo để hỗ trợ đa nền tảng và tăng khả năng tương tác với người dùng.
    \item Hoàn thiện hệ thống giám sát, logging và phân tích dữ liệu nhằm hỗ trợ quản lý và ra quyết định hiệu quả hơn.
    \item Triển khai hệ thống trên môi trường cloud nhằm tăng khả năng mở rộng và tính sẵn sàng của hệ thống.
\end{itemize}


\begin{thebibliography}{99}

\bibitem{bowersox}
Bowersox, D. J., Closs, D. J., \& Cooper, M. B. (2013).
\textit{Supply Chain Logistics Management}.
McGraw-Hill Education.

\bibitem{chopra}
Chopra, S., \& Meindl, P. (2016).
\textit{Supply Chain Management: Strategy, Planning, and Operation}.
Pearson Education.

\bibitem{toth}
Toth, P., \& Vigo, D. (2014).
\textit{Vehicle Routing: Problems, Methods, and Applications}.
SIAM.

\bibitem{dantzig}
Dantzig, G. B., \& Ramser, J. H. (1959).
The Truck Dispatching Problem.
\textit{Management Science}, 6(1), 80--91.

\bibitem{laporte}
Laporte, G. (1992).
The Vehicle Routing Problem: An Overview of Exact and Approximate Algorithms.
\textit{European Journal of Operational Research}, 59(3), 345--358.

\bibitem{pressman}
Pressman, R. S., \& Maxim, B. R. (2019).
\textit{Software Engineering: A Practitioner’s Approach}.
McGraw-Hill Education.

\bibitem{somerville}
Somerville, I. (2016).
\textit{Software Engineering}.
Pearson Education.

\bibitem{database}
Silberschatz, A., Korth, H. F., \& Sudarshan, S. (2019).
\textit{Database System Concepts}.
McGraw-Hill Education.

\bibitem{googlemaps}
Google Developers. (2023).
\textit{Google Maps Platform Documentation}.
https://developers.google.com/maps

\bibitem{osm}
OpenStreetMap Contributors. (2023).
\textit{OpenStreetMap Documentation}.
https://www.openstreetmap.org

\end{thebibliography}

\clearpage

\begin{center}
  {\bfseries\fontsize{14pt}{16pt}\selectfont
  MỨC ĐỘ ĐÓNG GÓP CỦA THÀNH VIÊN TRONG NHÓM\par}
\end{center}

\vspace{0.5cm} % khoảng cách dưới tiêu đề

\begin{table}[H]
    \centering
    \renewcommand{\arraystretch}{1.3}
    \setlength{\tabcolsep}{8pt}
    \begin{tabular}{|c|p{4cm}|p{7cm}|c|}
        \hline
        \textbf{TT} &
        \textbf{Họ và tên thành viên} &
        \textbf{Các nội dung, công việc thành viên đóng góp cho nhóm} &
        \textbf{Mức độ đóng góp} \\
        \hline
        1 & Lê Tuấn Khang &  &  \\ \hline % sau mỗi dấu & lần lượt điền Các nội dung, công việc thành viên đóng góp cho nhóm và Mức độ đóng góp
        2 & Hồ Huỳnh Nhu &  &  \\ \hline
        3 & Phạm Ngọc Diễm My &  &  \\ \hline
        4 & Nguyễn Huỳnh Tường Vi &  &  \\ \hline
        5 & Huỳnh Sĩ Nguyên &  &  \\ \hline
        6 & Trương Thục Trinh &  &  \\ \hline
        7 & Nguyễn Hoàng Uyển Nhi &  &  \\ \hline
    \end{tabular}
\end{table}

