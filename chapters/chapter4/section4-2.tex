\section{Giao diện người dùng}

Phần này trình bày các giao diện chính của hệ thống giao nhận bưu phẩm. Các giao diện được thiết kế đơn giản, trực quan, phù hợp với từng vai trò người dùng gồm khách hàng, shipper và admin.

\subsection{Giao diện bắt đầu}

Giao diện bắt đầu là màn hình đầu tiên khi người dùng mở ứng dụng, có chức năng giới thiệu hệ thống và điều hướng người dùng đến các chức năng tiếp theo.

\begin{figure}[H]
    \centering
    \includegraphics[width=0.35\textwidth]{frontmatter/image/start.jpg}
    \caption{Giao diện bắt đầu}
    \label{fig:start}
\end{figure}

\subsection{Giao diện đăng nhập}

Giao diện đăng nhập cho phép người dùng nhập thông tin tài khoản để truy cập vào hệ thống.

\begin{figure}[H]
    \centering
    \includegraphics[width=0.35\textwidth]{frontmatter/image/login.jpg}
    \caption{Giao diện đăng nhập}
    \label{fig:login}
\end{figure}

\subsection{Giao diện đăng ký}

Giao diện đăng ký cho phép người dùng tạo tài khoản mới bằng cách nhập các thông tin cần thiết.

\begin{figure}[H]
    \centering
    \includegraphics[width=0.35\textwidth]{frontmatter/image/register.jpg}
    \caption{Giao diện đăng ký}
    \label{fig:register}
\end{figure}

\subsection{Giao diện hồ sơ cá nhân}

Giao diện hồ sơ cá nhân cho phép người dùng xem và cập nhật thông tin cá nhân.

\begin{figure}[H]
    \centering
    \includegraphics[width=0.35\textwidth]{frontmatter/image/profile.jpg}
    \caption{Giao diện hồ sơ cá nhân}
    \label{fig:profile}
\end{figure}

\subsection{Giao diện Dashboard}

Sau khi đăng nhập, người dùng được chuyển đến Dashboard tương ứng với vai trò của mình trong hệ thống.

\begin{figure}[H]
    \centering
    \includegraphics[width=0.8\textwidth]{frontmatter/image/dashboard.jpg}
    \caption{Giao diện Dashboard theo vai trò}
    \label{fig:dashboard}
\end{figure}

\section{Triển khai các chức năng chính}

Phần này mô tả quá trình triển khai các chức năng chính của hệ thống, bao gồm giao diện người dùng kết hợp với xử lý nghiệp vụ và lưu trữ dữ liệu.

\subsection{Triển khai chức năng dành cho khách hàng}

\subsubsection{Tạo đơn hàng}

Khách hàng có thể tạo đơn hàng mới bằng cách nhập thông tin lấy hàng và giao hàng.

\begin{figure}[H]
    \centering
    \includegraphics[width=0.4\textwidth]{frontmatter/image/taodon.jpg}
    \caption{Giao diện tạo đơn hàng}
    \label{fig:create_order}
\end{figure}

\subsubsection{Theo dõi trạng thái đơn hàng}

Chức năng theo dõi đơn hàng cho phép khách hàng quan sát trạng thái xử lý và giao hàng.

\begin{figure}[H]
    \centering
    \includegraphics[width=0.4\textwidth]{frontmatter/image/theodoidon.jpg}
    \caption{Giao diện theo dõi trạng thái đơn hàng}
    \label{fig:track_order}
\end{figure}

\subsection{Triển khai chức năng dành cho Admin}





\subsection{Triển khai chức năng dành cho Shipper}

\subsubsection{Danh sách đơn hàng}

Shipper có thể xem danh sách các đơn hàng được phân công.

\begin{figure}[H]
    \centering
    \includegraphics[width=0.4\textwidth]{frontmatter/image/donshipper.jpg}
    \caption{Danh sách đơn hàng của shipper}
    \label{fig:shipper_orders}
\end{figure}





\subsection{Triển khai chức năng xác thực}

Hệ thống sử dụng Supabase Authentication để thực hiện đăng ký, đăng nhập và quản lý phiên làm việc của người dùng.
Sau khi xác thực thành công, thông tin người dùng được liên kết với bảng \texttt{profiles}.

\subsection{Triển khai quản lý hồ sơ cá nhân}

Người dùng có thể xem và cập nhật thông tin cá nhân như họ tên, số điện thoại và ảnh đại diện.
Dữ liệu được lưu trữ và đồng bộ thông qua Supabase Database.

\subsection{Triển khai thuật toán tối ưu lộ trình}

Thuật toán tối ưu lộ trình được triển khai dưới dạng một microservice độc lập.
Dịch vụ này nhận danh sách các điểm giao hàng và trả về thứ tự giao hàng tối ưu nhằm giảm tổng quãng đường di chuyển.


