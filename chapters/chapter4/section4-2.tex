\section{Giao diện người dùng}

Phần này trình bày các giao diện chính của hệ thống giao nhận bưu phẩm. Các giao diện được thiết kế đơn giản, trực quan, phù hợp với từng vai trò người dùng gồm khách hàng, shipper và admin.


\subsection{Giao diện đăng nhập}

Giao diện đăng nhập cho phép người dùng nhập thông tin tài khoản để truy cập vào hệ thống.

\begin{figure}[H]
    \centering
    \includegraphics[width=1\textwidth]{frontmatter/image/login.jpg}
    \caption{Giao diện đăng nhập}
    \label{fig:login}
\end{figure}

Giao diện đăng nhập của hệ thống DeliveryDriver được thiết kế đơn giản, hiện đại với bố cục tập trung ở trung tâm màn hình. Giao diện gồm các trường Email, Mật khẩu, tùy chọn Ghi nhớ đăng nhập, liên kết Quên mật khẩu và nút Đăng nhập nổi bật. Ngoài ra, hệ thống hỗ trợ đăng nhập nhanh bằng Google và Facebook, cùng liên kết Đăng ký cho người dùng mới. Thiết kế trực quan, dễ sử dụng và thuận tiện cho người dùng.
\subsection{Giao diện đăng ký}



\begin{figure}[H]
    \centering
    \includegraphics[width=1\textwidth]{frontmatter/image/register.jpg}
    \caption{Giao diện đăng ký}
    \label{fig:register}
\end{figure}
Giao diện tạo tài khoản mới của hệ thống DeliveryDriver được thiết kế tối giản, bố cục tập trung ở trung tâm màn hình. Giao diện bao gồm các trường thông tin: Họ và tên, Email, Số điện thoại, Mật khẩu và Xác nhận mật khẩu. Nút Đăng ký giúp người dùng dễ dàng hoàn tất quá trình tạo tài khoản. Liên kết Đăng nhập dành cho người đã có tài khoản và tùy chọn quay về trang chủ, đảm bảo tính thuận tiện và dễ sử dụng.
\subsection{Giao diện hồ sơ cá nhân}


\begin{figure}[H]
    \centering
    \includegraphics[width=1\textwidth]{frontmatter/image/profile.jpg}
    \caption{Giao diện hồ sơ cá nhân}
    \label{fig:profile}
\end{figure}
Giao diện Thông tin cá nhân cho phép người dùng quản lý thông tin tài khoản và bảo mật. Màn hình hiển thị ảnh đại diện, họ tên và vai trò người dùng, kèm các chức năng Chỉnh sửa thông tin và Đổi mật khẩu. Phần nội dung chính bao gồm các thông tin cá nhân như Họ và tên, Email, Số điện thoại, Mã khu vực, Địa chỉ chi tiết và Tọa độ vị trí. Bố cục rõ ràng, dễ theo dõi, hỗ trợ người dùng cập nhật và quản lý thông tin hiệu quả.
\subsection{Giao diện Dashboard của khách hàng}


\begin{figure}[H]
    \centering
    \includegraphics[width=1\textwidth]{frontmatter/image/dashboard.jpg}
    \caption{Giao diện Dashboard của khách hàng}
    \label{fig:dashboard}
\end{figure}
Giao diện Dashboard khách hàng hiển thị tổng quan thông tin sử dụng hệ thống sau khi đăng nhập. Phần đầu màn hình chào người dùng và cung cấp thống kê nhanh gồm tổng đơn hàng, đơn đang xử lý và đơn đã hoàn thành. Bên dưới là nút Tạo đơn hàng mới giúp người dùng thao tác nhanh. Khu vực Truy cập nhanh cho phép chuyển đến các chức năng chính như Hồ sơ, Địa chỉ và Đơn hàng.

\subsection{Giao diện tạo đơn hàng của khách hàng}



\begin{figure}[H]
    \centering
    \includegraphics[width=1\textwidth]{frontmatter/image/taodon1.jpg}
    \
    \caption{Giao diện tạo đơn hàng - nhập thông tin người gửi}
    \label{fig:create_order-1}
\end{figure}
Giao diện Tạo đơn hàng mới – bước nhập thông tin được thiết kế theo dạng quy trình, giúp người dùng dễ dàng theo dõi các bước tạo đơn. Màn hình tập trung vào việc khai báo thông tin người gửi và địa điểm lấy hàng, bao gồm số điện thoại, địa chỉ, mã khu vực và tọa độ. Hệ thống tích hợp chức năng chọn địa chỉ qua bản đồ, hỗ trợ xác định vị trí chính xác, đảm bảo quá trình tạo đơn diễn ra nhanh và hiệu quả

\begin{figure}[H]
    \centering
   
    \includegraphics[width=1\textwidth]{frontmatter/image/taodonhapnguoigui.jpg}
 
    \caption{Giao diện tạo đơn hàng - nhập thông tin người nhận}
    \label{fig:create_order-2}
\end{figure}
Giao diện Tạo đơn hàng – nhập thông tin người nhận cho phép người dùng khai báo thông tin giao hàng. Hệ thống hỗ trợ tìm kiếm người nhận có sẵn hoặc nhập thủ công thông tin gồm tên, số điện thoại, địa chỉ giao hàng, mã khu vực và tọa độ. Chức năng chọn địa chỉ qua bản đồ giúp xác định vị trí chính xác, cho phép thêm điểm giao hàng bổ sung, phù hợp với các đơn hàng có nhiều điểm giao.


\begin{figure}[H]
    \centering
   
    \includegraphics[width=1\textwidth]{frontmatter/image/taodon2.jpg}
    \caption{Giao diện tạo đơn hàng -  chọn bưu cục và dịch vụ}
    \label{fig:create_order-3}
\end{figure}
Giao diện Tạo đơn hàng – bước chọn bưu cục và dịch vụ cho phép người dùng lựa chọn bưu cục xử lý theo khu vực thông qua chức năng lọc. Bên dưới, hệ thống cung cấp các phương thức giao hàng để người dùng lựa chọn, kèm theo thông tin chi phí dự kiến và thời gian giao hàng. Giao diện trình bày rõ ràng theo từng tùy chọn, hỗ trợ người dùng so sánh và lựa chọn phương án phù hợp trước khi xác nhận đơn hàng
\begin{figure}[H]
    \centering
    
    \includegraphics[width=1\textwidth]{frontmatter/image/taodon3.jpg}
    \caption{Giao diện tạo đơn hàng - xác nhận đơn hàng} 
    \label{fig:create_order-4}
\end{figure}

Giao diện Tạo đơn hàng – bước xác nhận hiển thị toàn bộ thông tin đơn hàng trước khi hoàn tất. Màn hình tổng hợp các nội dung gồm bưu cục xử lý, thông tin người gửi, thông tin người nhận và phương thức giao hàng đã chọn. Các thông tin được trình bày theo từng khối rõ ràng, giúp người dùng dễ dàng kiểm tra và xác nhận tính chính xác trước khi tạo đơn hàng.

\subsection{Giao diện theo dõi đơn hàng của khách hàng}



\begin{figure}[H]
    \centering
    \includegraphics[width=1\textwidth]{frontmatter/image/theodoidon.jpg}
    \caption{Giao diện theo dõi trạng thái đơn hàng}
    \label{fig:track_order}
\end{figure}
 
Giao diện theo dõi trạng thái đơn hàng hiển thị chi tiết thông tin và tiến trình xử lý của một đơn hàng cụ thể. Hiển thị mã đơn hàng, thời gian tạo và trạng thái hiện tại. Khu vực theo dõi cho biết các mốc như đơn hàng đã được tạo và đã được xác nhận. Ngoài ra, giao diện hiển thị thêm thông tin lấy hàng, điểm giao hàng và thông tin tổng quan của đơn, cho phép người dùng hủy đơn hàng hoặc quay lại danh sách đơn.

\subsection{Giao diện dashboard của admin}

%\subsubsection{Theo dõi trạng thái shipper}

%Admin có thể theo dõi tình trạng hoạt động của shipper trong hệ thống.

%\begin{figure}[H]
    % \centering
    % \includegraphics[width=0.8\textwidth]{frontmatter/image/trangthaishipper.jpg}
    % \caption{Giao diện theo dõi trạng thái shipper}
    % \label{fig:shipper_status}
% \end{figure}


    \begin{figure}[H]
        \centering
        \includegraphics[width=1\textwidth]{frontmatter/image/trangchuadmin.jpg}
        \caption{Giao diện trang chủ admin}
        \label{fig:manage_postoffice}
\end{figure}
Giao diện Dashboard Admin cung cấp cái nhìn tổng quan về tình trạng quản lý đơn hàng trong hệ thống. Màn hình hiển thị các chỉ số thống kê như tổng đơn hàng, đơn chờ duyệt, đã duyệt, đã hủy, đã lấy hàng và hoàn thành. Khu vực thao tác nhanh hỗ trợ truy cập nhanh các chức năng chính như quản lý đơn hàng, xử lý đơn khẩn cấp, phân công tài xế và duyệt đơn. Bên dưới là danh sách đơn hàng gần đây, giúp admin theo dõi và xử lý kịp thời.
\subsection{Giao diện quản lý đơn hàng của admin}

    
    \begin{figure}[H]
        \centering
        \includegraphics[width=1\textwidth]{frontmatter/image/thongkedonhang.jpg}
        
        \caption{Giao diện quản lí đơn hàng - xem tổng quát}
        \label{fig:order_statistics}
\end{figure}
 Giao diện Quản lý đơn hàng cho phép admin theo dõi và xử lý đơn theo từng trạng thái từ chờ duyệt đến đã lấy hàng. Màn hình hiển thị số lượng đơn theo trạng thái, kèm khu vực thống kê tổng quan như tổng đơn hàng, số đơn đã lấy hàng và tỷ lệ lấy hàng.




    \begin{figure}[H]
        \centering
        \includegraphics[width=1\textwidth]{frontmatter/image/thongkedonhang1.jpg}
        \caption{Giao diện quản lí đơn hàng - xem chi tiết}
        \label{fig:order_statistics-detail}
\end{figure}
Giao diện Quản lý đơn hàng – xem chi tiết hiển thị số lượng đơn theo từng trạng thái xử lý gồm: chờ duyệt, chờ lấy hàng, đã lên lịch lấy, đã lấy hàng, hoàn thành và đã hủy. Hiển thị theo danh sách giúp admin dễ dàng theo dõi tiến độ, kiểm soát luồng xử lý và phát hiện các trạng thái tồn đọng.

 \subsection{Giao diện quản lý bưu cục của admin}
    Admin quản lí thông tin bưu cục bao gồm thêm, sửa, xóa bưu cục.
    \begin{figure}[H]
        \centering
        \includegraphics[width=1\textwidth]{frontmatter/image/quanlybuucuc.jpg}
        \caption{Giao diện quản lý bưu cục}
        \label{fig:manage_postoffice}
\end{figure}


\subsection{Giao diện admin duyệt đơn hàng với dịch vụ gửi thẳng cho bưu cục}
    Admin có thể xem danh sách đơn hàng đã gửi thẳng cho bưu cục.
    \begin{figure}[H]
        \centering
        \includegraphics[width=1\textwidth]{frontmatter/image/donhangguithrecthbuucuc.jpg}
        \caption{Giao diện duyệt đơn hàng gửi thẳng cho bưu cục}
        \label{fig:manage_postoffice-2}
\end{figure}


\subsection{Giao diện admin duyệt đơn hàng với dịch vụ bưu cục đến nhà lấy hàng}
    Admin có thể xem danh sách đơn hàng bưu cục đến nhà lấy hàng.
    \begin{figure}[H]
        \centering
        \includegraphics[width=\textwidth]{frontmatter/image/donhangbuucuctonha.jpg}
        \caption{Giao diện duyệt đơn hàng bưu cục đến nhà lấy hàng}
        \label{fig:manage_postoffice-3}
\end{figure}



\subsubsection{Xử lý và phân công đơn hàng đột xuất }



 \begin{figure}[H]
     \centering
     \includegraphics[width=1\textwidth]{frontmatter/image/xulydon.jpg}
     \caption{Giao diện xử lý và phân công đơn hàng đột xuất}
     \label{fig:assign_order}
 \end{figure}

%\subsubsection{Lên lịch giao hàng}

%Admin có thể lên lịch giao hàng cho shipper dựa trên tình trạng hoạt động.

% \begin{figure}[H]
%     \centering
%     \includegraphics[width=0.8\textwidth]{frontmatter/image/lichshipper.jpg}
%     \caption{Giao diện lên lịch cho shipper}
%     \label{fig:schedule}
% \end{figure}

\subsection{Giao diện dashboard của shipper}



\subsubsection{Danh sách đơn hàng}

Shipper có thể xem danh sách các đơn hàng được phân công.

 %\begin{figure}[H]
 %    \centering
  %   \includegraphics[width=0.8\textwidth]{frontmatter/image/donshipper.jpg}
   %  \caption{Danh sách đơn hàng của shipper}
   %  \label{fig:shipper_orders}
 %\end{figure}

\subsubsection{Màn hình bản đồ}

Màn hình bản đồ hỗ trợ shipper theo dõi vị trí và lộ trình giao hàng.

% \begin{figure}[H]
%     \centering
%     \includegraphics[width=0.4\textwidth]{frontmatter/image/mapshipper.jpg}
%     \caption{Màn hình bản đồ shipper}
%     \label{fig:map}
% \end{figure}



\subsection{Triển khai chức năng xác thực}

Hệ thống sử dụng Supabase Authentication để thực hiện đăng ký, đăng nhập và quản lý phiên làm việc của người dùng.
Sau khi xác thực thành công, thông tin người dùng được liên kết với bảng \texttt{profiles}.

\subsection{Triển khai quản lý hồ sơ cá nhân}

Người dùng có thể xem và cập nhật thông tin cá nhân như họ tên, số điện thoại và ảnh đại diện.
Dữ liệu được lưu trữ và đồng bộ thông qua Supabase Database.

\subsection{Triển khai thuật toán tối ưu lộ trình}

Thuật toán tối ưu lộ trình được triển khai dưới dạng một microservice độc lập.
Dịch vụ này nhận danh sách các điểm giao hàng và trả về thứ tự giao hàng tối ưu nhằm giảm tổng quãng đường di chuyển.


