\section{Giao diện người dùng}

Phần này trình bày các giao diện chính của hệ thống giao nhận bưu phẩm. Các giao diện được thiết kế đơn giản, trực quan, phù hợp với từng vai trò người dùng gồm khách hàng, shipper và admin.

\subsection{Giao diện bắt đầu}

Giao diện bắt đầu là màn hình đầu tiên khi người dùng mở ứng dụng, có chức năng giới thiệu hệ thống và điều hướng người dùng đến các chức năng tiếp theo.

\begin{figure}[H]
    \centering
    \includegraphics[width=0.8\textwidth]{frontmatter/image/start.jpg}
    \caption{Giao diện bắt đầu}
    \label{fig:start}
\end{figure}

Giao diện tạo tài khoản mới của hệ thống \textbf{DeliveryDriver} được thiết kế theo hướng tối giản, với bố cục tập trung ở trung tâm màn hình nhằm giúp người dùng dễ dàng thao tác và theo dõi. Giao diện bao gồm các trường thông tin cơ bản như: \textit{Họ và tên}, \textit{Email}, \textit{Số điện thoại}, \textit{Mật khẩu} và \textit{Xác nhận mật khẩu}, đáp ứng đầy đủ yêu cầu cho quá trình đăng ký tài khoản.

Nút \textbf{Đăng ký} được bố trí rõ ràng, cho phép người dùng nhanh chóng hoàn tất việc tạo tài khoản. Bên cạnh đó, giao diện còn cung cấp liên kết \textbf{Đăng nhập} dành cho người dùng đã có tài khoản, cùng với tùy chọn quay về trang chủ, đảm bảo tính thuận tiện, thân thiện và nâng cao trải nghiệm người dùng.

\subsection{Giao diện đăng nhập}

Giao diện đăng nhập cho phép người dùng nhập thông tin tài khoản để truy cập vào hệ thống.

\begin{figure}[H]
    \centering
    \includegraphics[width=0.8\textwidth]{frontmatter/image/login.jpg}
    \caption{Giao diện đăng nhập}
    \label{fig:login}
\end{figure}

Giao diện tạo tài khoản mới của hệ thống \textbf{DeliveryDriver} được thiết kế tối giản, với bố cục tập trung ở trung tâm màn hình. Giao diện bao gồm các trường thông tin: \textit{Họ và tên}, \textit{Email}, \textit{Số điện thoại}, \textit{Mật khẩu} và \textit{Xác nhận mật khẩu}. Nút \textbf{Đăng ký} giúp người dùng dễ dàng hoàn tất quá trình tạo tài khoản. Liên kết \textbf{Đăng nhập} dành cho người đã có tài khoản và tùy chọn quay về trang chủ, đảm bảo tính thuận tiện và dễ sử dụng.


\subsection{Giao diện đăng ký}

Giao diện đăng ký cho phép người dùng tạo tài khoản mới bằng cách nhập các thông tin cần thiết.

\begin{figure}[H]
    \centering
    \includegraphics[width=0.8\textwidth]{frontmatter/image/register.jpg}
    \caption{Giao diện đăng ký}
    \label{fig:register}
\end{figure}

\subsection{Giao diện hồ sơ cá nhân}

Giao diện hồ sơ cá nhân cho phép người dùng xem và cập nhật thông tin cá nhân.

\begin{figure}[H]
    \centering
    \includegraphics[width=0.8\textwidth]{frontmatter/image/profile.jpg}
    \caption{Giao diện hồ sơ cá nhân}
    \label{fig:profile}
\end{figure}

Giao diện \textbf{Thông tin cá nhân} cho phép người dùng quản lý thông tin tài khoản và bảo mật. Màn hình hiển thị ảnh đại diện, họ tên và vai trò người dùng, kèm theo các chức năng \textbf{Chỉnh sửa thông tin} và \textbf{Đổi mật khẩu}. Phần nội dung chính bao gồm các thông tin cá nhân như \textit{Họ và tên}, \textit{Email}, \textit{Số điện thoại}, \textit{Mã khu vực}, \textit{Địa chỉ chi tiết} và \textit{Tọa độ vị trí}. Bố cục giao diện rõ ràng, dễ theo dõi, hỗ trợ người dùng cập nhật và quản lý thông tin một cách hiệu quả.


\subsection{Giao diện Dashboard}

Sau khi đăng nhập, người dùng được chuyển đến Dashboard tương ứng với vai trò của mình trong hệ thống.

\begin{figure}[H]
    \centering
    \includegraphics[width=0.8\textwidth]{frontmatter/image/dashboard.jpg}
    \includegraphics[width=0.8\textwidth]{frontmatter/image/dashboard1.jpg}
    \caption{Giao diện Dashboard theo vai trò}
    \label{fig:dashboard}
\end{figure}

Giao diện \textbf{Dashboard khách hàng} hiển thị tổng quan thông tin sử dụng hệ thống sau khi người dùng đăng nhập. Phần đầu màn hình hiển thị lời chào người dùng và cung cấp các thống kê nhanh, bao gồm tổng số đơn hàng, số đơn đang xử lý và số đơn đã hoàn thành. Bên dưới là nút \textbf{Tạo đơn hàng mới} giúp người dùng thực hiện thao tác nhanh chóng. Khu vực \textbf{Truy cập nhanh} cho phép người dùng chuyển đến các chức năng chính như \textit{Hồ sơ}, \textit{Địa chỉ} và \textit{Đơn hàng}.


\section{Triển khai các chức năng chính}

Phần này mô tả quá trình triển khai các chức năng chính của hệ thống, bao gồm giao diện người dùng kết hợp với xử lý nghiệp vụ và lưu trữ dữ liệu.

\subsection{Triển khai chức năng dành cho khách hàng}

\subsubsection{Tạo đơn hàng}

Khách hàng có thể tạo đơn hàng mới bằng cách nhập thông tin lấy hàng và giao hàng.

\begin{figure}[H]
    \centering
    \includegraphics[width=0.8\textwidth]{frontmatter/image/taodon1.jpg}
    \includegraphics[width=0.8\textwidth]{frontmatter/image/taodon2.jpg}
    \includegraphics[width=0.8\textwidth]{frontmatter/image/taodonhapnguoigui.jpg}
    \includegraphics[width=0.8\textwidth]{frontmatter/image/taodon3.jpg}
    \caption{Giao diện tạo đơn hàng}
    \label{fig:create_order}
\end{figure}

\subsubsection{Theo dõi trạng thái đơn hàng}

Chức năng theo dõi đơn hàng cho phép khách hàng quan sát trạng thái xử lý và giao hàng.

\begin{figure}[H]
    \centering
    \includegraphics[width=0.8\textwidth]{frontmatter/image/theodoidon.jpg}
    \caption{Giao diện theo dõi trạng thái đơn hàng}
    \label{fig:track_order}
\end{figure}

Giao diện \textbf{Tạo đơn hàng mới – bước nhập thông tin} được thiết kế theo dạng quy trình, giúp người dùng dễ dàng theo dõi các bước tạo đơn. Màn hình tập trung vào việc khai báo thông tin người gửi và địa điểm lấy hàng, bao gồm \textit{số điện thoại}, \textit{địa chỉ}, \textit{mã khu vực} và \textit{tọa độ}. Hệ thống tích hợp chức năng chọn địa chỉ thông qua bản đồ, hỗ trợ xác định vị trí chính xác, đảm bảo quá trình tạo đơn hàng diễn ra nhanh chóng và hiệu quả.


\subsection{Triển khai chức năng dành cho Admin}

%\subsubsection{Theo dõi trạng thái shipper}

%Admin có thể theo dõi tình trạng hoạt động của shipper trong hệ thống.

%\begin{figure}[H]
    % \centering
    % \includegraphics[width=0.8\textwidth]{frontmatter/image/trangthaishipper.jpg}
    % \caption{Giao diện theo dõi trạng thái shipper}
    % \label{fig:shipper_status}
% \end{figure}

\subsubsection{trang chủ admin}
    Admin sẽ được chuyển đến trang chủ admin sau khi đăng nhập thành công.
    Ở đây admin có thể theo dõi tổng quan về số lượng đơn hàng, ship
    \begin{figure}[H]
        \centering
        \includegraphics[width=0.8\textwidth]{frontmatter/image/trangchuadmin.jpg}
        \caption{Giao diện trang chủ admin}
        \label{fig:manage_postoffice}
\end{figure}

 \subsubsection{quản lý bưu cục}
    Admin quản lí thông tin bưu cục bao gồm thêm, sửa, xóa bưu cục.
    \begin{figure}[H]
        \centering
        \includegraphics[width=0.8\textwidth]{frontmatter/image/quanlybuucuc.jpg}
        \caption{Giao diện quản lý bưu cục}
        \label{fig:manage_postoffice}
\end{figure}

\subsubsection{đơn hàng đã gửi thẳng cho bưu cục}
    Admin có thể xem danh sách đơn hàng đã gửi thẳng cho bưu cục.
    \begin{figure}[H]
        \centering
        \includegraphics[width=0.8\textwidth]{frontmatter/image/donhangguithrecthbuucuc.jpg}
        \caption{Giao diện đơn hàng gửi thẳng cho bưu cục}
        \label{fig:manage_postoffice}
\end{figure}

\subsubsection{bưu cục đến nhà lấy hàng}
    Admin có thể xem danh sách đơn hàng bưu cục đến nhà lấy hàng.
    \begin{figure}[H]
        \centering
        \includegraphics[width=0.8\textwidth]{frontmatter/image/donhangbuucuctonha.jpg}
        \caption{Giao diện đơn hàng bưu cục đến nhà lấy hàng}
        \label{fig:manage_postoffice}
\end{figure}


\subsubsection{Quản lí đơn hàng}

    Admin có thể xem quản lí số liệu đơn hàng theo ngày, tuần, tháng.
    \begin{figure}[H]
        \centering
        \includegraphics[width=0.8\textwidth]{frontmatter/image/thongkedonhang.jpg}
        \includegraphics[width=0.8\textwidth]{frontmatter/image/thongkedonhang1.jpg}
        \caption{Giao diện quản lí số liệu đơn hàng}
        \label{fig:order_statistics}
\end{figure}

\subsubsection{Xử lý và phân công đơn hàng đột xuất }


Admin thực hiện phân công đơn hàng và xử lý các đơn khẩn cấp.

 \begin{figure}[H]
     \centering
     \includegraphics[width=0.8\textwidth]{frontmatter/image/xulydon.jpg}
     \caption{Giao diện xử lý và phân công đơn hàng đột xuất}
     \label{fig:assign_order}
 \end{figure}


%\subsubsection{Lên lịch giao hàng}

%Admin có thể lên lịch giao hàng cho shipper dựa trên tình trạng hoạt động.

% \begin{figure}[H]
%     \centering
%     \includegraphics[width=0.8\textwidth]{frontmatter/image/lichshipper.jpg}
%     \caption{Giao diện lên lịch cho shipper}
%     \label{fig:schedule}
% \end{figure}

\subsection{Triển khai chức năng dành cho Shipper}



\subsubsection{Danh sách đơn hàng}

Shipper có thể xem danh sách các đơn hàng được phân công.

 %\begin{figure}[H]
 %    \centering
  %   \includegraphics[width=0.8\textwidth]{frontmatter/image/donshipper.jpg}
   %  \caption{Danh sách đơn hàng của shipper}
   %  \label{fig:shipper_orders}
 %\end{figure}

\subsubsection{Màn hình bản đồ}

Màn hình bản đồ hỗ trợ shipper theo dõi vị trí và lộ trình giao hàng.

% \begin{figure}[H]
%     \centering
%     \includegraphics[width=0.4\textwidth]{frontmatter/image/mapshipper.jpg}
%     \caption{Màn hình bản đồ shipper}
%     \label{fig:map}
% \end{figure}



\subsection{Triển khai chức năng xác thực}

Hệ thống sử dụng Supabase Authentication để thực hiện đăng ký, đăng nhập và quản lý phiên làm việc của người dùng.
Sau khi xác thực thành công, thông tin người dùng được liên kết với bảng \texttt{profiles}.

\subsection{Triển khai quản lý hồ sơ cá nhân}

Người dùng có thể xem và cập nhật thông tin cá nhân như họ tên, số điện thoại và ảnh đại diện.
Dữ liệu được lưu trữ và đồng bộ thông qua Supabase Database.

\subsection{Triển khai thuật toán tối ưu lộ trình}

Thuật toán tối ưu lộ trình được triển khai dưới dạng một microservice độc lập.
Dịch vụ này nhận danh sách các điểm giao hàng và trả về thứ tự giao hàng tối ưu nhằm giảm tổng quãng đường di chuyển.


