\subsection{Khái niệm tối ưu chi phí}

Tối ưu chi phí là quá trình tổ chức, kiểm soát và phân bổ các nguồn lực tài chính một cách hợp lý nhằm giảm thiểu chi phí phát sinh nhưng vẫn đảm bảo hiệu quả hoạt động, chất lượng sản phẩm/dịch vụ và mục tiêu của tổ chức. Việc tối ưu chi phí không đồng nghĩa với cắt giảm chi phí một cách máy móc, mà hướng đến sử dụng chi phí hiệu quả và bền vững.

Trong bối cảnh cạnh tranh ngày càng gay gắt, tối ưu chi phí đóng vai trò quan trọng trong việc nâng cao năng lực cạnh tranh, gia tăng lợi nhuận và đảm bảo sự phát triển lâu dài của doanh nghiệp.

\subsection{Khái niệm tối ưu chi phí vận tải}

Tối ưu chi phí vận tải là quá trình lập kế hoạch, tổ chức và quản lý hoạt động vận chuyển hàng hóa nhằm giảm thiểu các chi phí liên quan đến vận tải như chi phí nhiên liệu, nhân công, phương tiện, bảo dưỡng, thời gian và các chi phí phát sinh khác, trong khi vẫn đảm bảo hàng hóa được giao đúng thời gian, đúng địa điểm và an toàn.

Tối ưu chi phí vận tải là một nội dung quan trọng trong quản lý logistics, góp phần làm giảm tổng chi phí chuỗi cung ứng và nâng cao hiệu quả hoạt động kinh doanh của doanh nghiệp.