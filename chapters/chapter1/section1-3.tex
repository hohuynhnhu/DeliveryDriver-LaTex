
\subsection{Thuật toán tối ưu hóa}

Hệ thống áp dụng nhiều thuật toán và kỹ thuật khác nhau nhằm tối ưu hóa cả tuyến đường giao hàng và lịch làm việc của tài xế:

\begin{itemize}

    \item \textbf{Nearest Neighbor}: Tạo tuyến đường ban đầu nhanh chóng
    \item \textbf{2-opt}: Cải thiện tuyến đường bằng cách loại bỏ crossings
    \item \textbf{OSRM API}: Tính toán tuyến đường thực tế trên bản đồ OpenStreetMap
\end{itemize}

Kết hợp cả hai thuật toán giúp giảm 10-30\% tổng quãng đường so với cách phân công thủ công.

\subsubsection{OSRM (Open Source Routing Machine)}

OSRM là routing engine mã nguồn mở cho OpenStreetMap, cung cấp các API để tính toán tuyến đường.

\textbf{Các API chính:}
\begin{itemize}
    \item \textbf{Route Service}: Tính tuyến đường nhanh nhất giữa các điểm
    \item \textbf{Table Service}: Ma trận khoảng cách giữa nhiều điểm
    \item \textbf{Match Service}: Map matching GPS traces lên đường
    \item \textbf{Trip Service}: Giải quyết Traveling Salesman Problem
\end{itemize}

\textbf{Ưu điểm so với Google Maps:}
\begin{itemize}
    \item Miễn phí, không giới hạn requests
    \item Có thể tự host server
    \item Dữ liệu OpenStreetMap cập nhật liên tục
    \item API đơn giản, response nhanh
\end{itemize}

\textbf{Ứng dụng trong dự án:}
\begin{itemize}
    \item Tính khoảng cách và thời gian thực tế giữa các điểm giao hàng
    \item Lấy tọa độ waypoints để vẽ tuyến đường trên Leaflet map
    \item Tối ưu thứ tự giao hàng với Trip Service
    \item Hiển thị hướng dẫn turn-by-turn cho tài xế
\end{itemize}
\begin{itemize}
    \item \textbf{Nearest Neighbor}: Tạo lời giải ban đầu nhanh chóng cho bài toán TSP.
    \item \textbf{2-opt}: Cải thiện tuyến đường bằng cách loại bỏ các đoạn cắt nhau (crossing).
    \item \textbf{OR-Tools (Google)}: Tối ưu thứ tự ghé thăm các điểm giao hàng dựa trên ma trận khoảng cách.
    \item \textbf{Google Maps / OSRM API}: Tính khoảng cách và thời gian di chuyển thực tế theo đường giao thông.
    \item \textbf{Ramer--Douglas--Peucker}: Đơn giản hóa dữ liệu GPS để giảm số lượng điểm lưu trữ.
    \item \textbf{Genetic Algorithm (GA)}: Tối ưu bài toán xếp lịch ca làm việc cho tài xế.
\end{itemize}

\subsubsection{Đơn giản hóa tuyến đường bằng thuật toán Ramer--Douglas--Peucker}

Trong quá trình ghi nhận lộ trình di chuyển của tài xế, dữ liệu GPS có thể chứa hàng nghìn điểm, gây tốn dung lượng lưu trữ và làm chậm quá trình xử lý. Hệ thống sử dụng thuật toán \textbf{Ramer--Douglas--Peucker (RDP)} để:

\begin{itemize}
    \item Giảm số lượng điểm trong polyline
    \item Vẫn giữ nguyên hình dạng tổng thể của tuyến đường
    \item Giảm dung lượng lưu trữ trong cơ sở dữ liệu
\end{itemize}

Thuật toán hoạt động bằng cách loại bỏ các điểm trung gian không cần thiết nếu sai lệch so với đoạn thẳng nối hai đầu nhỏ hơn ngưỡng $\varepsilon$.

\subsubsection{Tối ưu hóa lịch tài xế bằng Genetic Algorithm}

Bài toán xếp lịch tài xế là một bài toán tổ hợp (combinatorial optimization) với nhiều ràng buộc:

\begin{itemize}
    \item Giới hạn số giờ làm việc mỗi tài xế
    \item Phân bố đơn hàng theo khu vực
    \item Cân bằng tải công việc giữa các tài xế
\end{itemize}

Hệ thống sử dụng Genetic Algorithm (GA) để:

\begin{itemize}
    \item Mã hóa mỗi phương án xếp lịch như một cá thể (chromosome)
    \item Áp dụng các toán tử: selection, crossover, mutation
    \item Tối ưu dần hàm mục tiêu: giảm tổng thời gian giao hàng và tăng mức độ cân bằng tải
\end{itemize}

\subsubsection{Pipeline tối ưu hóa tuyến đường}

Quy trình xử lý và tối ưu tuyến đường được mô tả như sau:

\begin{itemize}
    \item Đầu vào: danh sách các điểm giao hàng (latitude, longitude)
    \item Tính ma trận khoảng cách bằng:
    \begin{itemize}
        \item \textbf{OSRM}: ưu tiên dùng để lấy khoảng cách theo đường thực tế
        \item \textbf{Haversine}: dùng làm phương án dự phòng (fallback)
    \end{itemize}
    \item Sử dụng OR-Tools để tối ưu thứ tự ghé thăm các điểm
    \item Sau khi có thứ tự tối ưu, gọiOSRM get\_route để vẽ tuyến đường thực tế trên bản đồ
\end{itemize}

\subsubsection{Sơ đồ luồng xử lý tối ưu tuyến}
\begin{figure}[H]
\centering
\begin{tikzpicture}[
    node distance=1.4cm,
    every node/.style={draw, rectangle, rounded corners, align=center, minimum width=3cm, minimum height=0.9cm},
    arrow/.style={->, thick}
]

\node (loc) {List locations\\(lat, lon)};
\node (dist) [below of=loc] {DistanceMatrixService};
\node (osrm) [below left of=dist, xshift=-2cm] {OSRM\\(ưu tiên)};
\node (hav) [below right of=dist, xshift=2cm] {Haversine\\(fallback)};
\node (matrix) [below of=dist, yshift=-1.2cm] {distance\_matrix};
\node (opt) [below of=matrix] {RouteOptimizer\\(OR-Tools)};
\node (order) [below of=opt] {Thứ tự đi tối ưu};
\node (route) [below of=order] {OSRM get\_route\\(vẽ đường thực)};

\draw[arrow] (loc) -- (dist);
\draw[arrow] (dist) -- (osrm);
\draw[arrow] (dist) -- (hav);
\draw[arrow] (osrm) -- (matrix);
\draw[arrow] (hav) -- (matrix);
\draw[arrow] (matrix) -- (opt);
\draw[arrow] (opt) -- (order);
\draw[arrow] (order) -- (route);

\end{tikzpicture}
\caption{Pipeline tối ưu hóa tuyến đường giao hàng}
\label{fig:routing_pipeline}
\end{figure}

