\section{Sequence Diagram}

\subsection{Sequence Diagram đăng ký tài khoản}

\begin{figure}[H]
    \centering
    \includegraphics[height=0.4\textheight]{frontmatter/image/Dangkytaikhoansequence.jpg}
    \caption{Sequence Diagram đăng ký tài khoản}
    \label{fig:dangkytaikhoan_seq}
\end{figure}

Sơ đồ mô tả quy trình đăng ký tài khoản người dùng mới trong hệ thống.
Người dùng nhập thông tin đăng ký thông qua ứng dụng Mobile App, sau đó hệ thống Backend API tiếp nhận và gửi yêu cầu tạo người dùng đến Supabase Auth.
Trong trường hợp email đã tồn tại, hệ thống sẽ trả về lỗi trùng lặp; ngược lại, nếu thông tin hợp lệ, hệ thống tiến hành tạo tài khoản mới, lưu trữ hồ sơ người dùng vào PostgreSQL và sử dụng Email Service để gửi thư xác thực.
Quy trình hoàn tất khi hệ thống thông báo đăng ký thành công và yêu cầu người dùng kiểm tra email để kích hoạt tài khoản.

\subsection{Sequence Diagram đăng nhập hệ thống}

\begin{figure}[H]
    \centering
    \includegraphics[height=0.4\textheight]{frontmatter/image/Dangnhaphethongsequence.jpg}
    \caption{Sequence Diagram đăng nhập hệ thống}
    \label{fig:dangnhaphethong_seq}
\end{figure}

Sơ đồ mô tả quy trình xác thực và đăng nhập vào hệ thống.
Người dùng nhập Email và Mật khẩu thông qua ứng dụng Mobile App, sau đó hệ thống gửi thông tin đến Backend API để thực hiện xác thực với Supabase.
Trong trường hợp thông tin đăng nhập sai, hệ thống trả về lỗi xác thực; ngược lại, nếu thông tin hợp lệ, hệ thống tiếp tục truy vấn PostgreSQL để lấy thông tin chi tiết người dùng (Vai trò và Trạng thái).
Quy trình kiểm tra tiếp trạng thái tài khoản: nếu bị khóa sẽ thông báo vô hiệu hóa, nếu hoạt động bình thường sẽ thiết lập phiên làm việc và chuyển hướng người dùng vào màn hình chính.

\subsection{Sequence Diagram tạo đơn hàng mới}

\begin{figure}[H]
    \centering
    \includegraphics[height=0.4\textheight]{frontmatter/image/Taodonhangmoisequence.jpg}
    \caption{Sequence Diagram tạo đơn hàng mới}
    \label{fig:taodonhangmoi_seq}
\end{figure}

Sơ đồ mô tả quy trình khách hàng khởi tạo một yêu cầu giao hàng mới trên hệ thống.
Khách hàng nhập thông tin người gửi và người nhận trên ứng dụng Customer App, sau đó Backend API thực hiện kiểm tra địa chỉ và tính toán khoảng cách thông qua Map Service để xác nhận khả năng phục vụ.
Sau khi địa chỉ được xác thực, khách hàng nhập thông tin hàng hóa và gửi yêu cầu tạo đơn. Hệ thống tiến hành lưu trữ bản ghi đơn hàng mới vào PostgreSQL, khởi tạo Mã vận đơn và đặt trạng thái ban đầu là ``Chờ xử lý''.
Quy trình kết thúc khi hệ thống thông báo tạo đơn thành công và chuyển hướng khách hàng đến màn hình chi tiết đơn hàng vừa tạo.

\subsection{Sequence Diagram xếp lịch và phân công}

\begin{figure}[H]
    \centering
    % Bạn hãy thay tên file ảnh tương ứng của bạn vào đây
    \includegraphics[height=0.6\textheight]{frontmatter/image/Phancongdonhangsequence.jpg}
    \caption{Sequence Diagram xếp lịch và phân công đơn hàng}
    \label{fig:xeplich_sequence}
\end{figure}

Sơ đồ mô tả quy trình Quản trị viên (Admin) thực hiện lập kế hoạch vận chuyển và phân công nhiệm vụ, kết hợp linh hoạt giữa cơ chế tối ưu hóa tự động của hệ thống và quyền kiểm soát thủ công của con người.

Quy trình bắt đầu khi Admin lựa chọn ngày và khu vực cần xếp lịch. Hệ thống Backend API truy vấn dữ liệu từ PostgreSQL để hiển thị Dashboard. Tại đây, Admin có thể kích hoạt tính năng "Chạy tự động", yêu cầu hệ thống tính toán và đề xuất lộ trình tối ưu dựa trên các thuật toán nội bộ trước khi trả về kết quả dự kiến.

Trong giai đoạn tinh chỉnh (kéo thả đơn hàng hoặc đổi tài xế), hệ thống thực hiện cơ chế kiểm tra ràng buộc thời gian thực. Mỗi thao tác đều được đối chiếu với dữ liệu về tải trọng và thời gian làm việc.
Nếu phát hiện vi phạm (như vượt quá tải trọng xe), hệ thống hiển thị cảnh báo rủi ro. Điểm đặc biệt là hệ thống cho phép Admin quyền quyết định: hoặc hủy thao tác để chọn lại, hoặc xác nhận Gán đè để cưỡng chế phân công trong các trường hợp ưu tiên.

Cuối cùng, khi Admin xác nhận lưu lịch trình, hệ thống ghi nhận dữ liệu chính thức vào cơ sở dữ liệu. Đồng thời, Backend API kích hoạt Notification Service để gửi thông báo nhiệm vụ đến từng tài xế trong danh sách và cập nhật trạng thái của toàn bộ đơn hàng thành sẵn sàng lấy.
\subsection{Sequence Diagram thực hiện lấy hàng}

\begin{figure}[H]
    \centering
    \includegraphics[height=0.4\textheight]{frontmatter/image/Thuchienlayhangsequence.jpg}
    \caption{Sequence Diagram thực hiện lấy hàng}
    \label{fig:thuchienlayhang_seq}
\end{figure}

Sơ đồ mô tả quy trình tài xế thực hiện nhận hàng hóa tại điểm lấy.
Khi tài xế đến nơi, họ xác nhận trên ứng dụng Driver App, sau đó Backend API ghi nhận thời gian, vị trí vào PostgreSQL và gửi thông báo cho khách hàng qua Notification Service.
Tiếp theo, tài xế kiểm tra hàng hóa, chụp ảnh bằng chứng và gửi xác nhận lên hệ thống. Tại đây quy trình rẽ nhánh: nếu hàng hóa đúng quy định, hệ thống lưu ảnh, cập nhật trạng thái đơn hàng sang ``Đang giao hàng'' và thông báo cho khách.
Ngược lại, trong trường hợp hàng hóa sai lệch hoặc hư hỏng, hệ thống ghi nhận sự cố, gửi cảnh báo đến Admin và hủy nhiệm vụ lấy hàng hiện tại.

\subsection{Sequence Diagram thực hiện giao hàng}

\begin{figure}[H]
    \centering
    \includegraphics[height=0.4\textheight]{frontmatter/image/Thuchiengiaohangsequence.jpg}
    \caption{Sequence Diagram thực hiện giao hàng}
    \label{fig:thuchiengiaohang_seq}
\end{figure}

Sơ đồ mô tả quy trình tài xế thực hiện các bước cuối cùng để giao kiện hàng cho người nhận.
Quy trình bắt đầu khi tài xế thực hiện gọi điện cho người nhận qua Driver App; Backend API sẽ hỗ trợ kết nối bằng cách truy xuất số điện thoại ảo từ cơ sở dữ liệu để bảo mật thông tin.
Sau khi tương tác với khách, tài xế cập nhật kết quả giao hàng kèm hình ảnh bằng chứng lên hệ thống. Tại đây, quy trình rẽ nhánh dựa trên kết quả thực tế:
Trong trường hợp giao thành công, hệ thống lưu ảnh bằng chứng, cập nhật trạng thái đơn hàng thành ``Giao thành công'' và gửi thông báo đến khách hàng.
Ngược lại, nếu giao thất bại, hệ thống ghi nhận lý do, cập nhật trạng thái lỗi và gửi cảnh báo đến cả Admin lẫn khách hàng để xử lý quy trình hoàn hàng.

\subsection{Sequence Diagram xử lý sự cố đơn hàng}

\begin{figure}[H]
    \centering
    \includegraphics[height=0.4\textheight]{frontmatter/image/Xulysucodonhangsequence.jpg}
    \caption{Sequence Diagram xử lý sự cố đơn hàng}
    \label{fig:xulysucodonhang_seq}
\end{figure}

Sơ đồ mô tả quy trình quản trị viên giải quyết các vấn đề phát sinh đối với đơn hàng thông qua giao diện Web Admin.
Quản trị viên bắt đầu bằng việc tìm kiếm đơn hàng lỗi; hệ thống Backend API sẽ truy vấn PostgreSQL để lấy dữ liệu chi tiết hiển thị lên màn hình.
Sau khi quản trị viên chọn hành động xử lý, quy trình rẽ nhánh dựa trên loại sự cố:
Nếu khách hàng đổi địa chỉ, hệ thống cập nhật thông tin mới; trường hợp địa chỉ mới khác khu vực hoạt động, hệ thống sẽ hủy gán tài xế hiện tại để đưa đơn về trạng thái chờ tìm tài xế mới.
Đối với trường hợp xử lý giao thất bại hoặc hủy đơn, hệ thống cập nhật trạng thái tương ứng  vào cơ sở dữ liệu.
Quy trình kết thúc khi hệ thống gửi thông báo cập nhật đến các bên liên quan thông qua Notification Service và làm mới danh sách trên giao diện quản trị.

\subsection{Sequence Diagram xử lý đơn khẩn cấp}

\begin{figure}[H]
    \centering
    \includegraphics[height=0.4\textheight]{frontmatter/image/XulydonkhancapSequence.jpg}
    \caption{Sequence Diagram xử lý đơn khẩn cấp (Driver gặp tai nạn)}
    \label{fig:xulydonkhan_cap_seq}
\end{figure}

Sơ đồ mô tả quy trình hệ thống và các tác nhân phối hợp xử lý khi tài xế gặp tai nạn hoặc sự cố nghiêm trọng trong quá trình giao hàng. 
Quy trình được kích hoạt từ phía ứng dụng tài xế: Driver thực hiện chức năng \textit{Báo sự cố}, hệ thống Backend API tiếp nhận thông tin (đơn hàng, loại sự cố, vị trí hiện tại) và ghi nhận một bản ghi sự cố khẩn cấp trong cơ sở dữ liệu, đồng thời gửi thông báo khẩn đến quản trị viên thông qua kênh Notification Service.

Trên giao diện Web Admin, quản trị viên mở chi tiết đơn khẩn cấp, xem thông tin đơn hàng, tài xế gặp nạn và vị trí xảy ra sự cố. 
Tiếp đó, Admin yêu cầu hệ thống đề xuất danh sách tài xế thay thế phù hợp; Backend API truy vấn cơ sở dữ liệu để tìm các tài xế đang rảnh trong khu vực lân cận và trả kết quả về cho Web Admin hiển thị.

Sau khi xem xét, quản trị viên chọn một tài xế mới và xác nhận thao tác gán lại đơn khẩn cấp.
Hệ thống cập nhật lại phân công đơn hàng cho tài xế mới trong cơ sở dữ liệu, đồng thời gửi thông báo đến tài xế mới và khách hàng về việc thay đổi tài xế giao hàng.
Quy trình kết thúc khi đơn khẩn cấp được gán thành công cho tài xế mới và trạng thái liên quan được đồng bộ trên hệ thống.


\subsection{Sequence Diagram tra cứu hành trình đơn}

\begin{figure}[H]
    \centering
    \includegraphics[height=0.4\textheight]{frontmatter/image/Tracuuhanhtrinhdonsequence.jpg}
    \caption{Sequence Diagram tra cứu hành trình đơn}
    \label{fig:tracuuhanhtrinhdon_seq}
\end{figure}

Sơ đồ mô tả quy trình khách hàng thực hiện kiểm tra trạng thái và lịch sử di chuyển của đơn hàng.
Khách hàng nhập Mã vận đơn trên giao diện Customer App và nhấn nút tra cứu, yêu cầu này được gửi đến Backend API.
Hệ thống truy vấn thông tin cơ bản từ PostgreSQL. Tại đây quy trình rẽ nhánh: nếu không tìm thấy đơn hàng, hệ thống báo lỗi mã không tồn tại cho người dùng.
Trong trường hợp đơn hàng hợp lệ, Backend API tiếp tục truy xuất danh sách các mốc thời gian chi tiết từ cơ sở dữ liệu.
Dữ liệu đầy đủ bao gồm thời gian tạo, thời gian lấy hàng và vị trí hiện tại được trả về để ứng dụng hiển thị dòng thời gian hành trình cho khách hàng.

\subsection{Sequence Diagram cập nhật trạng thái hoạt động}

\begin{figure}[H]
    \centering
    \includegraphics[height=0.4\textheight]{frontmatter/image/Capnhattrangthaihoatdongsequence.jpg}
    \caption{Sequence Diagram cập nhật trạng thái hoạt động}
    \label{fig:capnhattrangthai_seq}
\end{figure}

Sơ đồ mô tả quy trình tài xế thay đổi trạng thái làm việc (Rảnh, Bận, Hết ca...) trên ứng dụng Driver App để hệ thống điều phối đơn hàng phù hợp.
Tài xế chọn trạng thái mới từ menu, ứng dụng gửi yêu cầu đến Backend API. Quy trình xử lý rẽ nhánh đặc biệt khi tài xế chọn trạng thái ``Hết ca làm''.
Trong trường hợp này, hệ thống kiểm tra PostgreSQL xem còn đơn hàng chưa hoàn thành không. Nếu còn, hệ thống gửi cảnh báo lỗi yêu cầu hoàn tất; nếu không còn vướng bận, hệ thống cập nhật trạng thái nghỉ, chốt tổng kết ca và ghi nhận thời gian kết thúc.
Đối với các trạng thái thông thường khác (Rảnh, Bận), hệ thống thực hiện cập nhật trực tiếp vào cơ sở dữ liệu và gửi thông báo thay đổi thành công để ứng dụng hiển thị trạng thái hiện tại.

\subsection{Sequence Diagram Tối ưu hoá phân bổ lịch trình}

\begin{figure}[H]
    \centering
    \includegraphics[height=0.4\textheight]{frontmatter/image/ToiUuHoaPhanBoLichTrinh-SD.jpeg}
    \caption{Sequence Diagram tối ưu hoá phân bổ lịch trình}
    \label{fig:toiuuhoalichtrinh_seq}
\end{figure}

Sơ đồ trình tự mô tả quy trình tối ưu hóa lịch trình giao hàng tự động bắt đầu từ việc Quản trị viên gửi yêu cầu, hệ thống sẽ ngay lập tức truy vấn dữ liệu đơn hàng và tài xế để kiểm tra điều kiện cần; nếu dữ liệu hợp lệ, bộ điều phối gọi Thuật toán Di truyền (GA) để thực hiện hàng loạt vòng lặp tiến hóa nhằm tìm ra phương án phân bổ có điểm Fitness cao nhất dựa trên độ ưu tiên và quãng đường, sau đó kết quả được gán số thứ tự giao hàng, lưu trữ đồng bộ vào Supabase và phản hồi lịch trình hoàn chỉnh cho người dùng.