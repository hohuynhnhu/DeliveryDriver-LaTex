\section{Kiến trúc tổng thể hệ thống}

Hệ thống được xây dựng theo mô hình client--server và triển khai theo kiến trúc microservices,
nhằm đảm bảo khả năng mở rộng, dễ bảo trì và phân tách rõ ràng các chức năng nghiệp vụ.
Các dịch vụ backend không giao tiếp trực tiếp với frontend mà thông qua một API Gateway,
đóng vai trò là điểm truy cập duy nhất của hệ thống.

\textbf{Tầng trình diễn (Presentation Layer)} bao gồm:
\begin{itemize}
    \item \textbf{Ứng dụng di động (Mobile Application)}: được xây dựng theo mô hình MVVM,
    phục vụ cho Customer và Driver.
    \item \textbf{Ứng dụng web (Web Application)}: được xây dựng bằng Nuxt,
    phục vụ cho Admin và người dùng quản trị hệ thống.
\end{itemize}

\textbf{Tầng backend (Application Layer)} được triển khai theo kiến trúc microservices
sử dụng FastAPI, bao gồm các dịch vụ chính:
\begin{itemize}
    \item Authentication Service (Dịch vụ xác thực)
    \item Order Service (Dịch vụ quản lý đơn hàng)
    \item Driver Scheduling Service (Dịch vụ điều phối tài xế)
    \item Routing Optimization Service (Dịch vụ tối ưu lộ trình)
    \item Route Analysis Service (Dịch vụ phân tích tuyến đường)
    \item Incident Management Service (Dịch vụ xử lý sự cố)
    \item Notification Service (Dịch vụ thông báo)
\end{itemize}

Routing Optimization Service chịu trách nhiệm tính toán tuyến đường tối ưu dựa trên
dữ liệu bản đồ và khoảng cách thực tế từ OSRM.
Route Analysis Service thực hiện phân tích lịch sử di chuyển, đánh giá hiệu quả tuyến đường
(thời gian, quãng đường, độ trễ) nhằm hỗ trợ tối ưu vận hành.

Incident Management Service chịu trách nhiệm tiếp nhận, quản lý và xử lý các sự cố phát sinh
trong quá trình giao hàng như tai nạn, hỏng phương tiện hoặc thay đổi tài xế,
đồng thời phối hợp với các dịch vụ khác để cập nhật trạng thái hệ thống.

Tất cả các service backend giao tiếp với nhau thông qua FAST API và được định tuyến
thông qua API Gateway, giúp tập trung xử lý xác thực, phân quyền và kiểm soát luồng dữ liệu.

\textbf{Tầng dữ liệu (Data Layer)} sử dụng Supabase (PostgreSQL) để:
\begin{itemize}
    \item Lưu trữ dữ liệu nghiệp vụ (orders, drivers, routes, profiles, ...)
    \item Quản lý xác thực người dùng và phân quyền truy cập
    thông qua JWT và Row Level Security (RLS)
\end{itemize}

Ngoài ra, hệ thống tích hợp với các dịch vụ bản đồ mã nguồn mở:
\begin{itemize}
    \item \textbf{OpenStreetMap (OSM)}: cung cấp dữ liệu bản đồ
    \item \textbf{OSRM}: hỗ trợ tính toán khoảng cách và truy vấn tuyến đường thực tế
\end{itemize}



\begin{figure}[H]
\centering
\begin{tikzpicture}[
    node distance=1cm,
    every node/.style={
        draw, rectangle, rounded corners, align=center,
        minimum width=3cm, minimum height=0.9cm
    },
    arrow/.style={->, semithick}
]

% ===== Clients =====
\node (customer) {Customer App\\Ứng dụng khách hàng\\(Mobile -- MVVM)};
\node (driver) [below=of customer] {Driver App\\Ứng dụng tài xế\\(Mobile -- MVVM)};
\node (admin) [below=of driver] {Admin Web\\Trang quản trị\\(Nuxt)};

% ===== API Gateway =====
\node (gateway) [right=1.4cm of driver] {API Gateway\\Cổng giao tiếp API};

% ===== Backend Services =====
\node (auth)     [right=1.6cm of gateway, yshift=3.8cm]
{Authentication Service\\Dịch vụ xác thực};

\node (order)    [below=of auth]
{Order Service\\Dịch vụ đơn hàng};

\node (schedule) [below=of order]
{Driver Scheduling Service\\Dịch vụ điều phối tài xế};

\node (routing)  [below=of schedule]
{Routing Optimization Service\\Dịch vụ tối ưu lộ trình};

\node (analysis) [below=of routing]
{Route Analysis Service\\Dịch vụ phân tích tuyến đường};

\node (incident) [below=of analysis]
{Incident Management Service\\Dịch vụ xử lý sự cố};

\node (noti)     [below=of incident]
{Notification Service\\Dịch vụ thông báo};

% ===== Database =====
\node (db) [right=1.4cm of schedule]
{Supabase\\PostgreSQL + Auth};

% ===== External APIs =====
\node (osrm) [below=of db]
{OSRM / OSM\\Dịch vụ bản đồ};

% ===== Connections =====
\draw[arrow] (customer) -- (gateway);
\draw[arrow] (driver) -- (gateway);
\draw[arrow] (admin) -- (gateway);

\draw[arrow] (gateway) -- (auth);
\draw[arrow] (gateway) -- (order);
\draw[arrow] (gateway) -- (schedule);
\draw[arrow] (gateway) -- (routing);
\draw[arrow] (gateway) -- (analysis);
\draw[arrow] (gateway) -- (incident);
\draw[arrow] (gateway) -- (noti);

\draw[arrow] (auth) -- (db);
\draw[arrow] (order) -- (db);
\draw[arrow] (schedule) -- (db);
\draw[arrow] (routing) -- (db);
\draw[arrow] (analysis) -- (db);
\draw[arrow] (incident) -- (db);
\draw[arrow] (noti) -- (db);

\draw[arrow] (routing) -- (osrm);
\draw[arrow] (analysis) -- (osrm);

\end{tikzpicture}
\caption{Kiến trúc tổng thể hệ thống giao hàng theo mô hình microservices}
\label{fig:system_architecture}
\end{figure}

\section{Usecase Diagram }

\subsection{Các tác nhân của hệ thống}

Hệ thống có các tác nhân chính bao gồm:
\begin{itemize}
    \item Khách hàng (Customer)
    \item Quản lý vận hành (Admin)
    \item Nhân viên giao hàng (Driver)
    \item Hệ thống (System)
\end{itemize}

\subsection{Use Case tổng quát}
\begin{figure}[H]
    \centering
    \includegraphics[height=0.5\textheight]{frontmatter/image/hethong.jpg}
    \caption{Use case Diagram tổng quát}
    \label{fig:intro}
\end{figure}
\subsection{Actors của hệ thống}

Các actor trong hệ thống giao nhận hàng bao gồm:

\begin{itemize}
    \item \textbf{Customer}: Người dùng cuối sử dụng ứng dụng để tạo đơn giao hàng, lựa chọn hình thức gửi hàng (gửi tại bưu cục hoặc yêu cầu lấy hàng), theo dõi hành trình đơn hàng và xem lịch sử các thông báo liên quan đến đơn giao.
    
    \item \textbf{Driver}: Nhân viên giao nhận trực tiếp thực hiện việc lấy hàng và giao hàng. Driver sử dụng hệ thống để xem lịch phân công, cập nhật trạng thái đơn hàng, chia sẻ vị trí theo thời gian thực, xác nhận đã lấy hàng, đã giao hàng và báo cáo các sự cố hoặc trường hợp giao hàng thất bại.
    
    \item \textbf{Admin}: Người quản trị hệ thống chịu trách nhiệm điều phối và giám sát toàn bộ hoạt động giao nhận. Admin thực hiện các chức năng như phân công đơn hàng cho tài xế, giám sát hoạt động của các driver, tối ưu lộ trình và xử lý các vấn đề phát sinh liên quan đến đơn hàng.
    
    \item \textbf{User}: Actor tổng quát đại diện cho các loại người dùng trong hệ thống. User thực hiện các chức năng chung như đăng ký tài khoản, đăng nhập hệ thống, quản lý hồ sơ cá nhân và quản lý sổ địa chỉ.
\end{itemize}

Dựa trên yêu cầu hệ thống, các use case chính bao gồm:
\begin{itemize}
    \item Xác thực và quản lý hồ sơ người dùng
    \item Đặt lịch và quản lý đơn hàng
    \item Tối ưu lộ trình giao hàng
    \item Theo dõi vị trí và thông báo
\end{itemize}



\subsection{Use case Diagram chi tiết về xử lý đăng nhập đăng ký và hồ sơ}

\begin{figure}[H]
    \centering
    \includegraphics[height=0.4\textheight]{frontmatter/image/dangkidangnhap.jpg}
    \caption{Use case Diagram người dùng đăng nhập đăng ký và hồ sơ}
    \label{fig:intro}
\end{figure}
Dưới đây là các đặc tả use case của người dùng đăng nhập đăng ký và hồ sơ:

\begin{table}[H]
\centering
\caption{UC-01: Đăng nhập hệ thống}
\renewcommand{\arraystretch}{1.3}
\begin{tabular}{|p{4cm}|p{10cm}|}
\hline
\textbf{Use Case ID} & UC-01 \\ \hline
\textbf{Tên Use Case} & Đăng nhập hệ thống \\ \hline
\textbf{Actor} & Customer, Driver, Admin \\ \hline
\textbf{Mô tả} & Cho phép người dùng truy cập hệ thống bằng tài khoản đã đăng ký. Hệ thống xác thực, cấp quyền và chuyển hướng đến màn hình làm việc tương ứng. \\ \hline
\textbf{Tiền điều kiện} & 
1. Tài khoản đã được kích hoạt.\\
& 2. Thiết bị có kết nối Internet. \\ \hline
\textbf{Luồng chính} &
1. Actor mở ứng dụng, chọn Đăng nhập.\\
& 2. Nhập Email hoặc Số điện thoại và Mật khẩu.\\
& 3. Nhấn nút "Đăng nhập".\\
& 4. Hệ thống kiểm tra xác thực.\\
& 5. Hệ thống lấy thông tin hồ sơ.\\
& 6. Thông báo đăng nhập thành công và chuyển hướng vào trang chủ. \\ \hline
\textbf{Luồng phụ} &
2a. Đăng nhập Google: Chọn Google $\rightarrow$ Xác thực $\rightarrow$ Vào hệ thống.\\
& 4a. Sai thông tin: Báo lỗi "Tài khoản hoặc mật khẩu không đúng".\\
& 4b. Chưa kích hoạt: Yêu cầu kiểm tra email kích hoạt. \\ \hline
\textbf{Hậu điều kiện} &
Thành công: Người dùng truy cập được hệ thống và lưu phiên làm việc.\\
& Thất bại: Vẫn ở màn hình đăng nhập. \\ \hline
\end{tabular}
\end{table}

\begin{table}[H]
\centering
\caption{UC-02: Đăng ký tài khoản}
\renewcommand{\arraystretch}{1.3}
\begin{tabular}{|p{4cm}|p{10cm}|}
\hline
\textbf{Use Case ID} & UC-02 \\ \hline
\textbf{Actor} & Customer \\ \hline
\textbf{Mô tả} & Người dùng mới tạo tài khoản để sử dụng dịch vụ. Hệ thống tự động tạo hồ sơ và gửi email xác thực. \\ \hline
\textbf{Tiền điều kiện} & Email chưa từng được sử dụng trong hệ thống. \\ \hline
\textbf{Luồng chính} &
1. Chọn "Đăng ký".\\
& 2. Nhập Họ tên, Email, Mật khẩu, Số điện thoại.\\
& 3. Nhấn "Đăng ký".\\
& 4. Hệ thống kiểm tra dữ liệu hợp lệ.\\
& 5. Tạo tài khoản và hồ sơ người dùng.\\
& 6. Gửi email xác thực.\\
& 7. Thông báo kiểm tra email. \\ \hline
\textbf{Luồng phụ} &
4a. Email đã tồn tại: Báo lỗi và yêu cầu nhập email khác.\\
& 4b. Mật khẩu yếu: Yêu cầu tối thiểu 6 ký tự. \\ \hline
\textbf{Hậu điều kiện} & Tài khoản được tạo ở trạng thái chờ xác thực. \\ \hline
\end{tabular}
\end{table}

\begin{table}[H]
\centering
\caption{UC-03: Quên mật khẩu}
\renewcommand{\arraystretch}{1.3}
\begin{tabular}{|p{4cm}|p{10cm}|}
\hline
\textbf{Actor} & Customer, Driver, Admin \\ \hline
\textbf{Mô tả} & Hỗ trợ người dùng đặt lại mật khẩu mới thông qua email. \\ \hline
\textbf{Tiền điều kiện} & Tài khoản đã tồn tại trong hệ thống. \\ \hline
\textbf{Luồng chính} &
1. Chọn "Quên mật khẩu".\\
& 2. Nhập Email đã đăng ký.\\
& 3. Nhấn "Gửi yêu cầu".\\
& 4. Hệ thống gửi link đặt lại mật khẩu.\\
& 5. Actor mở email và truy cập link.\\
& 6. Nhập mật khẩu mới và xác nhận.\\
& 7. Hệ thống cập nhật và thông báo thành công. \\ \hline
\textbf{Luồng phụ} &
2a. Sai định dạng email: Yêu cầu nhập lại.\\
& 5a. Link hết hạn: Yêu cầu gửi lại yêu cầu mới. \\ \hline
\textbf{Hậu điều kiện} & Mật khẩu mới có hiệu lực, mật khẩu cũ bị hủy. \\ \hline
\end{tabular}
\end{table}

\begin{table}[H]
\centering
\caption{UC-04: Xem hồ sơ cá nhân}
\renewcommand{\arraystretch}{1.3}
\begin{tabular}{|p{4cm}|p{10cm}|}
\hline
\textbf{Use Case ID} & UC-04 \\ \hline
\textbf{Actor} & Customer, Driver, Admin \\ \hline
\textbf{Mô tả} & Người dùng xem lại thông tin cá nhân của mình trên hệ thống. \\ \hline
\textbf{Tiền điều kiện} & Người dùng đã đăng nhập thành công. \\ \hline
\textbf{Luồng chính} &
1. Chọn menu "Tài khoản" hoặc nhấn vào ảnh đại diện.\\
& 2. Hệ thống tải dữ liệu hồ sơ từ máy chủ.\\
& 3. Hiển thị thông tin: ảnh đại diện, họ tên, số điện thoại, email, địa chỉ.\\
& 4. Hiển thị thêm trạng thái hoạt động nếu actor là Driver. \\ \hline
\textbf{Luồng phụ} &
2a. Lỗi mạng: Thông báo lỗi kết nối và hiển thị nút "Thử lại". \\ \hline
\textbf{Hậu điều kiện} & Thông tin cá nhân được hiển thị đầy đủ cho người dùng. \\ \hline
\end{tabular}
\end{table}

\begin{table}[H]
\centering
\caption{UC-05: Cập nhật hồ sơ}
\renewcommand{\arraystretch}{1.3}
\begin{tabular}{|p{4cm}|p{10cm}|}
\hline
\textbf{Use Case ID} & UC-05 \\ \hline
\textbf{Actor} & Customer, Driver, Admin \\ \hline
\textbf{Mô tả} & Cho phép người dùng chỉnh sửa thông tin cá nhân và thay đổi ảnh đại diện. \\ \hline
\textbf{Tiền điều kiện} & Người dùng đang ở màn hình xem hồ sơ cá nhân. \\ \hline
\textbf{Luồng chính} &
1. Nhấn nút "Chỉnh sửa".\\
& 2. Thay đổi các thông tin văn bản.\\
& 3. Chọn ảnh đại diện mới.\\
& 4. Nhấn "Lưu thay đổi".\\
& 5. Hệ thống tải ảnh và cập nhật dữ liệu.\\
& 6. Thông báo "Cập nhật thành công". \\ \hline
\textbf{Luồng phụ} &
3a. Ảnh không hợp lệ: Báo lỗi kích thước hoặc định dạng.\\
& 5a. Lỗi lưu dữ liệu: Thông báo lỗi hệ thống và yêu cầu thử lại. \\ \hline
\textbf{Hậu điều kiện} & Thông tin mới được lưu và hiển thị ngay lập tức. \\ \hline
\end{tabular}
\end{table}

\begin{table}[H]
\centering
\caption{UC-06: Đổi mật khẩu}
\renewcommand{\arraystretch}{1.3}
\begin{tabular}{|p{4cm}|p{10cm}|}
\hline
\textbf{Use Case ID} & UC-06 \\ \hline
\textbf{Actor} & Customer, Driver, Admin \\ \hline
\textbf{Mô tả} & Cho phép người dùng chủ động thay đổi mật khẩu nhằm tăng cường bảo mật. \\ \hline
\textbf{Tiền điều kiện} & Người dùng đang đăng nhập hệ thống. \\ \hline
\textbf{Luồng chính} &
1. Vào Cài đặt $\rightarrow$ Chọn "Đổi mật khẩu".\\
& 2. Nhập mật khẩu cũ, mật khẩu mới và xác nhận mật khẩu mới.\\
& 3. Nhấn "Xác nhận".\\
& 4. Hệ thống kiểm tra mật khẩu cũ.\\
& 5. Lưu mật khẩu mới và thông báo thành công. \\ \hline
\textbf{Luồng phụ} &
4a. Sai mật khẩu cũ: Báo lỗi và yêu cầu nhập lại.\\
& 2a. Mật khẩu xác nhận không khớp: Báo lỗi nhập liệu. \\ \hline
\textbf{Hậu điều kiện} & Mật khẩu cũ không còn hiệu lực, mật khẩu mới được áp dụng. \\ \hline
\end{tabular}
\end{table}

\begin{table}[H]
\centering
\caption{UC-07: Đăng xuất}
\renewcommand{\arraystretch}{1.3}
\begin{tabular}{|p{4cm}|p{10cm}|}
\hline
\textbf{Use Case ID} & UC-07 \\ \hline
\textbf{Actor} & Customer, Driver, Admin \\ \hline
\textbf{Mô tả} & Cho phép người dùng đăng xuất khỏi hệ thống và kết thúc phiên làm việc. \\ \hline
\textbf{Tiền điều kiện} & Người dùng đang đăng nhập hệ thống. \\ \hline
\textbf{Luồng chính} &
1. Chọn "Đăng xuất" từ menu.\\
& 2. Hệ thống yêu cầu xác nhận đăng xuất.\\
& 3. Người dùng chọn "Có".\\
& 4. Hệ thống xóa phiên đăng nhập trên thiết bị.\\
& 5. Chuyển về màn hình đăng nhập. \\ \hline
\textbf{Luồng phụ} &
3a. Chọn "Không": Hủy thao tác đăng xuất và quay lại hệ thống. \\ \hline
\textbf{Hậu điều kiện} & Ứng dụng trở về trạng thái chưa đăng nhập. \\ \hline
\end{tabular}
\end{table}

\subsection{Use case Diagram chi tiết đặt hàng và theo dõi}

\begin{figure}[H]
    \centering
    \includegraphics[height=0.4\textheight]{frontmatter/image/dathangtheodoi.jpg}
    \caption{Use case Diagram chi tiết đặt hàng và theo dõi}
    \label{fig:intro}
\end{figure}
Dưới đây là các đặc tả use case của chi tiết đặt hàng và theo dõi:

\begin{table}[H]
\centering
\caption{UC-08: Tạo đơn hàng mới}
\renewcommand{\arraystretch}{1.3}
\begin{tabular}{|p{4cm}|p{10cm}|}
\hline
\textbf{Use Case ID} & UC-08 \\ \hline
\textbf{Actor} & Customer \\ \hline
\textbf{Mô tả} & Người dùng tạo yêu cầu giao hàng mới. Người dùng có thể chọn hình thức bưu tá đến lấy hàng tại nhà hoặc tự mang hàng ra bưu cục. \\ \hline
\textbf{Tiền điều kiện} & Tài khoản đã đăng nhập vào hệ thống. \\ \hline
\end{tabular}
\end{table}

\begin{table}[H]
\centering
\caption{UC-09: Quản lý đơn hàng đã tạo}
\renewcommand{\arraystretch}{1.3}
\begin{tabular}{|p{4cm}|p{10cm}|}
\hline
\textbf{Use Case ID} & UC-09 \\ \hline
\textbf{Actor} & Customer \\ \hline
\textbf{Mô tả} & Người dùng xem danh sách các đơn hàng đã tạo, lọc theo trạng thái và xem chi tiết từng đơn hàng. Cho phép hủy đơn nếu đơn chưa được xử lý. \\ \hline
\textbf{Tiền điều kiện} & Người dùng đã đăng nhập vào hệ thống. \\ \hline
\textbf{Luồng chính} &
1. Actor truy cập menu "Đơn hàng của tôi".\\
& 2. Hệ thống hiển thị danh sách đơn hàng theo thời gian gần nhất.\\
& 3. Actor sử dụng bộ lọc để tìm đơn theo trạng thái xử lý.\\
& 4. Actor chọn một đơn hàng cụ thể để xem chi tiết.\\
& 5. Hệ thống hiển thị thông tin người gửi, người nhận, hàng hóa và lịch sử trạng thái. \\ \hline
\textbf{Luồng phụ} &
4a. Hủy đơn hàng: Nếu đơn ở trạng thái "Chờ xử lý", Actor chọn "Hủy đơn".\\
& \quad Hệ thống yêu cầu nhập lý do hủy và cập nhật trạng thái thành "Đã hủy".\\
& 2a. Không có dữ liệu: Hệ thống thông báo chưa có đơn hàng nào. \\ \hline
\textbf{Hậu điều kiện} & Người dùng nắm được tình trạng và chi tiết các đơn hàng của mình. \\ \hline
\end{tabular}
\end{table}

\begin{table}[H]
\centering
\caption{UC-010: Tra cứu hành trình đơn hàng}
\renewcommand{\arraystretch}{1.3}
\begin{tabular}{|p{4cm}|p{10cm}|}
\hline
\textbf{Use Case ID} & UC-10 \\ \hline
\textbf{Actor} & Customer \\ \hline
\textbf{Mô tả} & Cho phép người dùng theo dõi vị trí và trạng thái hiện tại của đơn hàng thông qua mã vận đơn. \\ \hline
\textbf{Tiền điều kiện} & Có mã vận đơn hợp lệ. \\ \hline
\textbf{Luồng chính} &
1. Actor truy cập trang "Tra cứu đơn hàng".\\
& 2. Nhập mã vận đơn vào ô tìm kiếm.\\
& 3. Nhấn nút "Tra cứu".\\
& 4. Hệ thống truy vấn dữ liệu từ lịch sử giao hàng.\\
& 5. Hiển thị timeline hành trình gồm thời gian tạo, lấy hàng, vị trí hiện tại và trạng thái xử lý.\\
& 6. Kết thúc tra cứu. \\ \hline
\textbf{Luồng phụ} &
2a. Sai mã vận đơn: Thông báo không tìm thấy thông tin đơn hàng.\\
& 4a. Đơn hàng chưa có cập nhật: Hiển thị trạng thái vừa tạo, chưa có hành trình di chuyển. \\ \hline
\textbf{Hậu điều kiện} & Thông tin hành trình đơn hàng được hiển thị cho người dùng. \\ \hline
\end{tabular}
\end{table}

\subsection{Use case Diagram chi tiết lấy hàng và giao hàng của tài xế}

\begin{figure}[H]
    \centering
    \includegraphics[height=0.3\textheight]{frontmatter/image/layhanggiaohang.jpg}
    \caption{Use case Diagram quản lý tài khoản và xác thực}
    \label{fig:intro}
\end{figure}
Dưới đây là các đặc tả use case lấy hàng và giao hàng của tài xế:

\begin{table}[H]
\centering
\caption{UC-11: Cập nhật trạng thái hoạt động}
\renewcommand{\arraystretch}{1.3}
\begin{tabular}{|p{4cm}|p{10cm}|}
\hline
\textbf{Use Case ID} & UC-11 \\ \hline
\textbf{Actor} & Driver \\ \hline
\textbf{Mô tả} & Tài xế chủ động thay đổi trạng thái làm việc để hệ thống xác định khả năng tiếp nhận đơn hàng mới. \\ \hline
\textbf{Tiền điều kiện} & Tài xế đã đăng nhập vào ứng dụng. \\ \hline
\textbf{Luồng chính} &
1. Actor truy cập màn hình trạng thái làm việc.\\
& 2. Hệ thống hiển thị trạng thái hiện tại.\\
& 3. Actor chọn trạng thái mới (Rảnh, Bận, Hết ca làm, Không hoạt động).\\
& 4. Hệ thống cập nhật trạng thái lên máy chủ.\\
& 5. Hệ thống điều chỉnh phân phối đơn hàng theo trạng thái mới.\\
& 6. Thông báo cập nhật thành công. \\ \hline
\textbf{Luồng phụ} &
3a. Chọn trạng thái Hết ca làm: Hệ thống kiểm tra đơn chưa hoàn thành và yêu cầu xử lý trước khi nghỉ.\\
& 4a. Mất kết nối mạng: Báo lỗi và giữ nguyên trạng thái cũ. \\ \hline
\textbf{Hậu điều kiện} & Trạng thái làm việc của tài xế được cập nhật trên toàn hệ thống. \\ \hline
\end{tabular}
\end{table}

\begin{table}[H]
\centering
\caption{UC-12: Xem lịch trình phân công}
\renewcommand{\arraystretch}{1.3}
\begin{tabular}{|p{4cm}|p{10cm}|}
\hline
\textbf{Use Case ID} & UC-12 \\ \hline
\textbf{Actor} & Driver \\ \hline
\textbf{Mô tả} & Tài xế xem danh sách các điểm lấy hàng và giao hàng được phân công theo lộ trình tối ưu. \\ \hline
\textbf{Tiền điều kiện} & Tài xế đang ở trạng thái sẵn sàng nhận việc. \\ \hline
\textbf{Luồng chính} &
1. Actor chọn menu "Lịch trình".\\
& 2. Hệ thống tải các chuyến xe được gán.\\
& 3. Hiển thị danh sách điểm dừng gồm địa chỉ, loại tác vụ và thời gian dự kiến.\\
& 4. Actor chọn một điểm dừng để xem chi tiết người liên hệ và hàng hóa.\\
& 5. Actor chuyển sang chế độ xem bản đồ để xem tuyến đường.\\
& 6. Kết thúc xem lịch trình. \\ \hline
\textbf{Luồng phụ} &
2a. Chưa có lịch phân công: Hiển thị thông báo chưa có nhiệm vụ mới.\\
& 3a. Lịch trình thay đổi từ Admin: Tự động làm mới và gửi thông báo cập nhật. \\ \hline
\textbf{Hậu điều kiện} & Tài xế nắm được lộ trình và thứ tự công việc cần thực hiện. \\ \hline
\end{tabular}
\end{table}

\begin{table}[H]
\centering
\caption{UC-13: Thực hiện lấy hàng}
\renewcommand{\arraystretch}{1.3}
\begin{tabular}{|p{4cm}|p{10cm}|}
\hline
\textbf{Use Case ID} & UC-13 \\ \hline
\textbf{Actor} & Driver \\ \hline
\textbf{Mô tả} & Tài xế đến địa chỉ người gửi để nhận hàng và xác nhận việc lấy hàng trên hệ thống. \\ \hline
\textbf{Tiền điều kiện} &
1. Tài xế đã nhận lịch trình.\\
& 2. Tài xế đã đến địa chỉ người gửi. \\ \hline
\textbf{Luồng chính} &
1. Actor chọn đơn hàng cần lấy trong lịch trình.\\
& 2. Actor nhấn "Bắt đầu lấy hàng".\\
& 3. Kiểm tra hàng hóa thực tế so với thông tin hệ thống.\\
& 4. Quét mã đơn hàng hoặc nhập mã xác nhận.\\
& 5. Nhấn "Xác nhận lấy hàng thành công".\\
& 6. Hệ thống cập nhật trạng thái "Đã lấy hàng".\\
& 7. Hệ thống đánh dấu nhiệm vụ đã hoàn thành. \\ \hline
\textbf{Luồng phụ} &
3a. Hàng hóa không hợp lệ: Từ chối nhận hàng và cập nhật lý do.\\
& 4a. Không liên lạc được người gửi: Ghi nhận trạng thái và xếp lịch lại. \\ \hline
\textbf{Hậu điều kiện} & Hàng hóa được chuyển sang trách nhiệm của tài xế. \\ \hline
\end{tabular}
\end{table}

\begin{table}[H]
\centering
\caption{UC-14: Thực hiện giao hàng}
\renewcommand{\arraystretch}{1.3}
\begin{tabular}{|p{4cm}|p{10cm}|}
\hline
\textbf{Use Case ID} & UC-14 \\ \hline
\textbf{Actor} & Driver \\ \hline
\textbf{Mô tả} & Tài xế giao hàng cho người nhận và ghi nhận bằng chứng giao hàng trên hệ thống. \\ \hline
\textbf{Tiền điều kiện} &
1. Tài xế đang giữ hàng hóa cần giao.\\
& 2. Tài xế đã đến địa chỉ người nhận. \\ \hline
\textbf{Luồng chính} &
1. Actor chọn đơn hàng cần giao.\\
& 2. Liên hệ người nhận để nhận hàng.\\
& 3. Giao kiện hàng cho người nhận.\\
& 4. Chụp ảnh bằng chứng hoặc yêu cầu ký nhận.\\
& 5. Nhấn "Hoàn tất giao hàng".\\
& 6. Hệ thống cập nhật trạng thái "Giao hàng thành công".\\
& 7. Đơn hàng được đóng lại. \\ \hline
\textbf{Luồng phụ} &
2a. Người nhận vắng nhà: Hẹn lại hoặc cập nhật giao không thành công.\\
& 3a. Người nhận từ chối: Ghi nhận lý do và cập nhật trạng thái chuyển hoàn. \\ \hline
\textbf{Hậu điều kiện} & Đơn hàng hoàn tất quy trình vận chuyển. \\ \hline
\end{tabular}
\end{table}

\subsection{Đặc tả Use Case Diagram Quản trị điều phối}

\begin{figure}[H]
    \centering
    % Đã sửa tên file ảnh theo yêu cầu
    \includegraphics[width=0.9\textwidth]{frontmatter/image/quantridieuphoi.jpg} 
    \caption{Use case Diagram quản trị điều phối}
    \label{fig:quantridieuphoi}
\end{figure}

Dưới đây là bảng đặc tả chi tiết cho các Use case quan trọng của hệ thống:

% --- UC-01: QUẢN LÝ BƯU CỤC ---
\begin{table}[H]
\centering
\caption{UC-15: Quản lý Bưu cục}
\renewcommand{\arraystretch}{1.3}
\begin{tabular}{|p{3.5cm}|p{11cm}|}
\hline
\textbf{Tên Use Case} & Quản lý Bưu cục \\ \hline
\textbf{Actor} & Admin \\ \hline
\textbf{Mô tả} & Quản trị viên thực hiện thêm mới, cập nhật thông tin, kích hoạt hoặc vô hiệu hóa các bưu cục (Post Office) để mở rộng hoặc thu hẹp mạng lưới vận chuyển. \\ \hline
\textbf{Tiền điều kiện} & Admin đã đăng nhập hệ thống. \\ \hline
\textbf{Luồng chính} &
\begin{enumerate}
    \item Actor truy cập danh sách bưu cục.
    \item Hệ thống hiển thị danh sách bưu cục cùng trạng thái hoạt động.
    \item Actor chọn chức năng "Thêm mới" hoặc chọn một bưu cục để "Chỉnh sửa".
    \item Actor nhập thông tin (Tên, Mã bưu cục, Địa chỉ, Mã vùng phụ trách).
    \item Actor thiết lập trạng thái (Kích hoạt/Vô hiệu hóa/Bảo trì).
    \item Hệ thống kiểm tra tính hợp lệ của dữ liệu (Mã không trùng lặp).
    \item Hệ thống lưu thông tin và thông báo thành công.
\end{enumerate} \\ \hline
\textbf{Luồng phụ} &
\textbf{6a. Mã bưu cục đã tồn tại:}
\begin{itemize}
    \item Hệ thống báo lỗi trùng mã định danh.
    \item Actor nhập lại mã mới.
\end{itemize} \\ \hline
\textbf{Hậu điều kiện} & Thông tin bưu cục được cập nhật vào cơ sở dữ liệu và có hiệu lực ngay lập tức. \\ \hline
\end{tabular}
\end{table}

% --- UC-16: QUẢN LÝ TÀI XẾ ---
\begin{table}[H]
\centering
\caption{UC-16: Quản lý Tài xế \& Theo dõi vị trí}
\renewcommand{\arraystretch}{1.3}
\begin{tabular}{|p{3.5cm}|p{11cm}|}
\hline
\textbf{Tên Use Case} & Quản lý Tài xế \\ \hline
\textbf{Actor} & Admin \\ \hline
\textbf{Mô tả} & Quản lý hồ sơ tài xế, cập nhật trạng thái làm việc (Sẵn sàng/Bận/Nghỉ) và theo dõi vị trí thực (Real-time tracking) trên bản đồ. \\ \hline
\textbf{Tiền điều kiện} & Có tài xế đã đăng ký trong hệ thống. \\ \hline
\textbf{Luồng chính} &
\begin{enumerate}
    \item Actor truy cập module Quản lý tài xế.
    \item Hệ thống hiển thị danh sách tài xế và trạng thái hiện tại.
    \item Actor chọn xem chi tiết một tài xế.
    \item Hệ thống hiển thị hồ sơ cá nhân, lịch sử đơn hàng và vị trí hiện tại trên bản đồ .
    \item Actor cập nhật trạng thái tài xế.
    \item Hệ thống lưu trạng thái mới.
\end{enumerate} \\ \hline
\textbf{Luồng phụ} &
\textbf{4a. Tài xế mất tín hiệu GPS:}
\begin{itemize}
    \item Hệ thống hiển thị cảnh báo "Mất kết nối" và thời gian cập nhật lần cuối.
\end{itemize}
\textbf{5a. Tài xế đang có đơn hàng chưa hoàn thành:}
\begin{itemize}
    \item Hệ thống ngăn chặn việc chuyển trạng thái sang "Inactive" hoặc "Off Duty".
    \item Hệ thống yêu cầu hoàn thành đơn hàng hoặc điều phối lại trước.
\end{itemize} \\ \hline
\textbf{Hậu điều kiện} & Trạng thái tài xế được đồng bộ để phục vụ cho việc xếp lịch. \\ \hline
\end{tabular}
\end{table}

% --- UC-17: THEO DÕI & XỬ LÝ ĐƠN ---
\begin{table}[H]
\centering
\caption{UC-17: Theo dõi \& Xử lý đơn}
\renewcommand{\arraystretch}{1.3}
\begin{tabular}{|p{3.5cm}|p{11cm}|}
\hline
\textbf{Tên Use Case} & Theo dõi \& Xử lý đơn \\ \hline
\textbf{Actor} & Admin \\ \hline
\textbf{Mô tả} & Quản lý vòng đời đơn hàng từ lúc tiếp nhận  đến khi hoàn tất, bao gồm việc phê duyệt, từ chối hoặc gom nhóm đơn theo vùng. \\ \hline
\textbf{Tiền điều kiện} & Có đơn hàng mới được tạo từ phía khách hàng . \\ \hline
\textbf{Luồng chính} &
\begin{enumerate}
    \item Actor xem danh sách đơn hàng mới .
    \item Actor kiểm tra thông tin chi tiết (địa chỉ, loại hàng).
    \item Actor thực hiện hành động "Phê duyệt"  để chuyển thành Order chính thức hoặc "Từ chối" .
    \item Nếu phê duyệt, hệ thống tự động gom nhóm đơn  dựa trên mã vùng .
    \item Hệ thống cập nhật trạng thái đơn hàng.
\end{enumerate} \\ \hline
\textbf{Luồng phụ} &
\textbf{3a. Từ chối đơn hàng:}
\begin{itemize}
    \item Actor nhập lý do từ chối (Sai địa chỉ, ...).
    \item Hệ thống gửi thông báo hủy đến khách hàng.
\end{itemize} \\ \hline
\textbf{Hậu điều kiện} & Đơn hàng chuyển sang trạng thái "Confirmed" và sẵn sàng để xếp lịch. \\ \hline
\end{tabular}
\end{table}

% --- UC-04: XẾP LỊCH THỦ CÔNG ---
\begin{table}[H]
\centering
\caption{UC-18: Xếp lịch thủ công}
\renewcommand{\arraystretch}{1.3}
\begin{tabular}{|p{3.5cm}|p{11cm}|}
\hline
\textbf{Tên Use Case} & Xếp lịch thủ công \\ \hline
\textbf{Actor} & Admin \\ \hline
\textbf{Mô tả} & Cho phép Admin tự tay chọn tài xế và gán các đơn hàng cụ thể vào lịch trình của tài xế đó, thường dùng cho các trường hợp đặc biệt hoặc xử lý sự cố. \\ \hline
\textbf{Tiền điều kiện} & Đơn hàng đã được xác nhận, Tài xế đang ở trạng thái sẵn sàng. \\ \hline
\textbf{Luồng chính} &
\begin{enumerate}
    \item Actor chọn ngày và khu vực cần xếp lịch.
    \item Hệ thống hiển thị danh sách đơn chưa gán và danh sách tài xế khả dụng.
    \item Actor kéo thả hoặc chọn các đơn hàng gán cho một tài xế cụ thể.
    \item Hệ thống tính toán tải trọng và cảnh báo nếu vượt quá giới hạn.
    \item Actor xác nhận "Tạo lịch trình" .
    \item Hệ thống lưu bản ghi vào bảng.
\end{enumerate} \\ \hline
\textbf{Luồng phụ} &
\textbf{4a. Tài xế quá tải:}
\begin{itemize}
    \item Hệ thống cảnh báo tổng khối lượng/số đơn vượt mức cho phép.
    \item Actor bỏ bớt đơn hoặc xác nhận gán đè.
\end{itemize} \\ \hline
\textbf{Hậu điều kiện} & Lịch trình được tạo và gửi đến App của tài xế. \\ \hline
\end{tabular}
\end{table}

% --- UC-05: XẾP LỊCH TỰ ĐỘNG (GA) ---
\begin{table}[H]
\centering
\caption{UC-19: Xếp lịch tự động (Genetic Algorithm)}
\renewcommand{\arraystretch}{1.3}
\begin{tabular}{|p{3.5cm}|p{11cm}|}
\hline
\textbf{Tên Use Case} & Xếp lịch tự động (GA) \\ \hline
\textbf{Actor} &Admin\\ \hline
\textbf{Mô tả} & Hệ thống sử dụng thuật toán di truyền (Genetic Algorithm) để tối ưu hóa lộ trình giao hàng, giảm thiểu quãng đường và cân bằng tải giữa các tài xế. \\ \hline
\textbf{Tiền điều kiện} & Danh sách đơn hàng lớn cần xử lý trong ngày. \\ \hline
\textbf{Luồng chính} &
\begin{enumerate}
    \item Actor chọn khu vực và nhấn "Tối ưu hóa lịch trình".
    \item Hệ thống thu thập dữ liệu đơn hàng và vị trí kho/bưu cục.
    \item Hệ thống khởi tạo quần thể (Population) các phương án lộ trình.
    \item Hệ thống chạy vòng lặp  để tìm phương án có Fitness Score cao nhất (Quãng đường ngắn nhất, thời gian nhanh nhất).
    \item Hệ thống đề xuất lịch trình tối ưu cho admin xem trước.
    \item admin xem xét và áp dụng.
\end{enumerate} \\ \hline
\textbf{Luồng phụ} &
\textbf{5a. Không tìm thấy phương án thỏa mãn:}
\begin{itemize}
    \item Do số lượng đơn quá lớn so với số tài xế.
    \item Hệ thống cảnh báo và đề xuất để lại một số đơn sang ca sau.
\end{itemize} \\ \hline
\textbf{Hậu điều kiện} & Hàng loạt lịch trình được tạo tự động nhanh chóng và tối ưu. \\ \hline
\end{tabular}
\end{table}

% --- UC-20: HỦY/CẬP NHẬT LỊCH ---
\begin{table}[H]
\centering
\caption{UC-20: Hủy/Cập nhật lịch trình}
\renewcommand{\arraystretch}{1.3}
\begin{tabular}{|p{3.5cm}|p{11cm}|}
\hline
\textbf{Tên Use Case} & Hủy/Cập nhật lịch \\ \hline
\textbf{Actor} & Admin \\ \hline
\textbf{Mô tả} & Chỉnh sửa lịch trình đang chạy hoặc chưa chạy (thêm/bớt đơn, đổi tài xế) hoặc hủy bỏ hoàn toàn lịch trình do sự cố. \\ \hline
\textbf{Tiền điều kiện} & Lịch trình đã tồn tại. \\ \hline
\textbf{Luồng chính} &
\begin{enumerate}
    \item Actor tìm kiếm mã lịch trình (Schedule ID).
    \item Actor chọn "Chỉnh sửa".
    \item Hệ thống cho phép gỡ bỏ đơn hàng khỏi lịch (đơn sẽ quay lại trạng thái Pending) hoặc chuyển lịch sang tài xế khác.
    \item Actor lưu thay đổi.
    \item Hệ thống gửi thông báo cập nhật cho tài xế liên quan.
\end{enumerate} \\ \hline
\textbf{Luồng phụ} &
\textbf{2a. Hủy lịch trình đang chạy:}
\begin{itemize}
    \item Tài xế đã bắt đầu đi giao.
    \item Hệ thống yêu cầu xác nhận khẩn cấp và gửi thông báo dừng ngay lập tức cho tài xế.
\end{itemize} \\ \hline
\textbf{Hậu điều kiện} & Lịch trình được cập nhật lại trạng thái và dữ liệu. \\ \hline
\end{tabular}
\end{table}

% --- UC-07: XEM DASHBOARD THỐNG KÊ ---
\begin{table}[H]
\centering
\caption{UC-21: Xem Dashboard thống kê}
\renewcommand{\arraystretch}{1.3}
\begin{tabular}{|p{3.5cm}|p{11cm}|}
\hline
\textbf{Tên Use Case} & Xem Dashboard thống kê \\ \hline
\textbf{Actor} & Admin \\ \hline
\textbf{Mô tả} & Cung cấp cái nhìn tổng quan về hiệu suất hệ thống: Tổng đơn hàng, Doanh thu, Tỷ lệ giao thành công, Hiệu suất tài xế. \\ \hline
\textbf{Tiền điều kiện} & Admin đăng nhập hệ thống. \\ \hline
\textbf{Luồng chính} &
\begin{enumerate}
    \item Actor truy cập trang chủ (Dashboard).
    \item Hệ thống truy vấn dữ liệu tổng hợp từ Database và Data Warehouse.
    \item Actor lọc dữ liệu theo khoảng thời gian (Ngày/Tuần/Tháng).
\end{enumerate} \\ \hline
\textbf{Hậu điều kiện} & Admin nắm bắt được tình hình kinh doanh để ra quyết định. \\ \hline
\end{tabular}
\end{table}

% --- UC-08: GỢI Ý MỞ BƯU CỤC MỚI ---
\begin{table}[H]
\centering
\caption{UC-22: Gợi ý mở bưu cục mới (Analytics)}
\renewcommand{\arraystretch}{1.3}
\begin{tabular}{|p{3.5cm}|p{11cm}|}
\hline
\textbf{Tên Use Case} & Gợi ý mở bưu cục mới \\ \hline
\textbf{Actor} & Admin  \\ \hline
\textbf{Mô tả} & Hệ thống sử dụng thuật toán phân tích giỏ hàng (Market Basket Analysis - Apriori) để phân tích tần suất các tuyến đường và gợi ý vị trí tiềm năng đặt bưu cục mới. \\ \hline
\textbf{Tiền điều kiện} & Dữ liệu lịch sử đơn hàng đủ lớn được lưu trong Database. \\ \hline
\textbf{Luồng chính} &
\begin{enumerate}
    \item Actor chọn module Phân tích nâng cao.
    \item Actor chọn chức năng "Gợi ý địa điểm".
    \item Actor thiết lập tham số (Độ hỗ trợ tối thiểu - Min Support).
    \item Hệ thống chạy thuật toán khai phá dữ liệu trên các cặp điểm giao nhận phổ biến.
    \item Hệ thống hiển thị danh sách các khu vực có nhu cầu cao nhưng chưa có bưu cục phủ sóng.
    \item Hệ thống hiển thị dự báo doanh thu nếu mở bưu cục tại đó.
\end{enumerate} \\ \hline
\textbf{Hậu điều kiện} & Admin có dữ liệu định lượng để quyết định chiến lược mở rộng. \\ \hline
\end{tabular}
\end{table}

